%%%% CAPÍTULO 1 - INTRODUÇÃO
%%
%% Deve apresentar uma visão global da pesquisa, incluindo: breve histórico, importância e justificativa da escolha do tema,
%% delimitações do assunto, formulação de hipóteses e objetivos da pesquisa e estrutura do trabalho.

%% Título e rótulo de capítulo (rótulos não devem conter caracteres especiais, acentuados ou cedilha)
\chapter{Introdução}\label{cap:introducao} % REVISAR


A sociedade contemporânea enfrenta desafios críticos relacionados à sustentabilidade e à gestão de seus recursos naturais. O crescente custo dos recursos hídricos e energéticos, somado à necessidade global de promover a eficiência energética e combater as mudanças climáticas, evidencia a importância de um consumo mais consciente no ambiente doméstico. O lar, historicamente visto como um espaço de consumo passivo, está no centro desta discussão, pois pequenas mudanças de hábito, multiplicadas por milhões de residências, têm um impacto agregado substancial.

No contexto brasileiro, essa preocupação é duplamente relevante. O país, apesar de sua vasta reserva hídrica, enfrenta paradoxalmente crises de escassez regionais e uma matriz elétrica majoritariamente composta por de usinas hidrelétricas. Isso cria uma interdependência crítica: em tempos de seca, a escassez de água não apenas ameaça o abastecimento, mas também encarece a geração de energia elétrica.

Dentro da residência, o chuveiro elétrico figura consistentemente como um dos principais vilões do consumo, podendo representar uma parcela significativa da fatura de eletricidade e ser um ponto de alto consumo de água.

Tendo em vista o cenario apresentado, o presente projeto foi desenvolvido com a finalidade de gerar um relatorio detalhado do consumo em chuveiros elétricos, conscientizando os usuários sobre o impacto de seus habitos, e levando-os a adotar habitos mais sustentaveis.

\section{Objetivos}

\subsection{Objetivos gerais} %revisar?

Este trabalho tem por objetivo primario o desenvolvimento um sistema IoT de baixo custo para monitoramento do consumo de água e energia em chuveiros elétricos. A partir do qual uma plataforma online é desenvolvida, fornecendo ao usuário dados em tempo real, e históricos de seus banhos.
E dessa forma, com esses dados registrados e expostos de maneira clara e acessivel, promover a conscientização sobre os custos e impactos de cada banho, incentivando a adoção de hábitos mais sustentáveis.

\subsection{Objetivos especificos}

\begin{itemize}
\item Medição e monitoramento de potência de sistemas de corrente alternada.
\item Estudo e desenvolvimeto de sistemas embarcados para IoT.
\item Desenvolvimento de uma interface web para gerir o sistema.
\end{itemize}

\section{Justificativa} %revisado
%%%%%%parte velha
%Dentro das mudanças de hábitos mais impactantes que um indivíduo pode tomar na esfera brasileira, destaca-se a busca por banhos mais curtos, que leva a um impacto menor tanto financeiro quanto ambiental.

%Porém, nota-se que certas atividades, como o banho, são dificilmente praticadas de forma que seu impacto é visível imediatamente. O usuário não tem uma noção das quantidades de água e de energia elétrica gastas em cada banho, nem os custos de cada minuto adicional. A única forma de feedback são as faturas mensais, o que dificulta para o usuário associar uma diminuição do tempo do banho com uma diminuição tangível das contas de água e energia.

%A partir disso, propõe-se o desenvolvimento e implementação de um sistema IoT para monitoramento e gestão do consumo de água e energia no banho, disponibilizando esses dados para o usuário por meio de um WebApp. O objetivo é que, com base nessas informações, o usuário se conscientize do impacto de suas ações, sendo oferecido tanto um feedback imediato quanto relatórios históricos, permitindo o acompanhamento da evolução do consumo e identificação de padrões no mesmo.
%%%%%%%
No Brasil, o chuveiro elétrico é um dos aparelhos eletrodomésticos mais comuns nas residências, de acordo com \cite{EPE:PDE2031:2022}, é presente em mais de 70\% dos lares. Sua popularidade se deve ao seu baixo custo e facilidade de instalação, que dispensa a complexa infraestrutura exigida por sistemas a gás ou solar. No entanto, esse aparelho é também um dos maiores responsáveis pelo consumo residencial de energia elétrica.

O consumo de energia elétrica do chuveiro é o terceiro maior em uma residência típica, ficando atrás apenas do ar-condicionado e da geladeira. Em média, o chuveiro elétrico responde por cerca de 14\% do consumo total de energia elétrica doméstica \cite{EPE:PDE2031:2022}. Um banho de 15 minutos em um chuveiro de 5,5 kW, por exemplo, consome cerca de 1,375 kWh, o que, multiplicado pelo valor médio da tarifa residencial no Brasil (aproximadamente R\$ 0,78/kWh, em 2025 \cite{ANEEL:RankingTarifas:2025}), representa um custo em torno de R\$ 1,07 por banho — valor que pode dobrar em regiões com bandeiras tarifárias vermelhas ou escassez hídrica. 

Além do impacto causado por seu alto consumo energético, o uso do chuveiro elétrico também está diretamente ligado ao consumo de água, de acordo com \cite{MARZALL_NASCIMENTO:2023} pode ser atribuido a 25,89\% da água consumida por uma residencia. Um chuveiro convencional pode consumir entre 4 e 6 litros de água por minuto \cite{NORTEAQUECEDORES:2018}, o que significa que um único banho de 15 minutos pode gastar até 180 litros de água. Esse volume representa uma parcela significativa do consumo diário per capita, estimado em 148,2 litros \cite{SNSA:SNIS_AE_2023}. 

A partir disso, propõe-se o desenvolvimento e implementação de um sistema IoT para monitoramento e gestão do consumo de água e energia no banho, disponibilizando esses dados para o usuário por meio de um WebApp. O objetivo é que, com base nessas informações, o usuário se conscientize do impacto de suas ações, sendo oferecido tanto um feedback imediato quanto relatórios históricos, permitindo o acompanhamento da evolução do consumo e identificação de padrões no mesmo.
%Portanto, o chuveiro elétrico exerce uma dupla pressão sobre os recursos domésticos: eleva os custos com energia elétrica e contribui substancialmente para o consumo de água potável.