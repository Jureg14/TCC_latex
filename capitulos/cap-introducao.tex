%%%% CAPÍTULO 1 - INTRODUÇÃO
%%
%% Deve apresentar uma visão global da pesquisa, incluindo: breve histórico, importância e justificativa da escolha do tema,
%% delimitações do assunto, formulação de hipóteses e objetivos da pesquisa e estrutura do trabalho.

%% Título e rótulo de capítulo (rótulos não devem conter caracteres especiais, acentuados ou cedilha)
\chapter{Introdução}\label{cap:introducao}

A sociedade contemporânea enfrenta desafios críticos relacionados à sustentabilidade e à gestão de seus recursos naturais. O crescente custo dos recursos hídricos e energéticos, somado à necessidade global de promover a eficiência energética e combater as mudanças climáticas, evidencia a importância de um consumo mais consciente no ambiente doméstico. O lar, historicamente visto como um espaço de consumo passivo, está no centro desta discussão, pois pequenas mudanças de hábito, multiplicadas por milhões de residências, têm um impacto agregado substancial.

No contexto brasileiro, essa preocupação é duplamente relevante. O país, apesar de sua vasta reserva hídrica, enfrenta paradoxalmente crises de escassez regionais e uma matriz elétrica majoritariamente composta por de usinas hidrelétricas. Isso cria uma interdependência crítica: em tempos de seca, a escassez de água não apenas ameaça o abastecimento, mas também encarece a geração de energia elétrica.

Dentro da residência, o chuveiro elétrico figura consistentemente como um dos principais vilões do consumo, podendo representar uma parcela significativa da fatura de eletricidade e ser um ponto de alto consumo de água.

Tendo em vista o cenario apresentado, o presente projeto foi desencolvido com a finalidade de gerar um relatorio detalhado do consumo em chuveiros elétricos, conscientizando os usuarios sobre o impacto de seus habitos, e levando-os a adotar habitos mais sustentaveis.


% Segundo \citeonline{Coulouris2013}.

% Segundo \citeonline[p. 40]{Coulouris2013}.

% Citação no final do Parágrafo~\cite{Coulouris2013}. 

% Citação no final do Parágrafo com número de página~\cite[p. 40]{Coulouris2013}.

% %(Modelo de referência: pessoa jurídica)
% Citação no final do Parágrafo~\cite{NBR6023:2018}

% %(Modelo de referência: pessoa jurídica)
% Citação no final do Parágrafo~\cite{NBR6027:2012}

% %(Modelo de referência: pessoa jurídica)
% Citação no final do Parágrafo~\cite{NBR6028:2021}

% Segundo a \citeonline{NBR14724:2011}.

% Citação no final do Parágrafo~\cite{NBR10520:2002}

% Citação no final do Parágrafo~\cite{NBR14724:2011}.

% % (Modelo de referência de trabalho acadêmico).
% Citação no final do Parágrafo~\cite{Andrade2005}

% % (Modelo de referência: capítulo de livro).
% Citação no final do Parágrafo~\cite{Borges2014}

% % (Modelo de referência: leis, decretos, portarias, etc.)
% Citação no final do Parágrafo~\cite{BRASIL:1998}

% % (Modelo de referência: livro com subtítulo). Nome com sufixo "Von" - Configuração no bib
% Citação no final do Parágrafo~\cite[p. 66]{KROGH:2001}

% Citação no final do Parágrafo~\cite{Faina2001}

% % (Modelo de referência: livro com subtítulo).
% Citação no final do Parágrafo~\cite{Davenport2012}

% % (Modelo de referência: artigo de periódico).
% Citação no final do Parágrafo~\cite{Monteiro2009}

% %(Modelo de referência: artigo de periódico). Nome familiar "Junior"
% Citação no final do Parágrafo~\cite{Sanches2024}

% % (Modelo de referência: trabalho publicado em evento).
% Citação no final do Parágrafo~\cite{Renaux2001}

\section{Objetivos}

\subsection{Objetivos gerais}

Este trabalho tem por objetivo desenvolver um sistema IoT de baixo custo para monitoramento do consumo de água e energia em chuveiros elétricos, fornecendo ao usuário dados em tempo real pela internet local, e dessa forma, promover a conscientização sobre os custos e impactos de cada banho, incentivando a adoção de hábitos mais sustentáveis.

\subsection{Objetivos especificos}

\begin{itemize}
\item Medição e monitoramento de potencia de sistemas de corrente alternada.
\item Estudo e desenvolvimeto de sistemas embarcados para IoT.
\item Desenvolvimento de uma interface web para gerir o sistema.
\end{itemize}

\section{Justificativa}

A busca por banhos mais curtos e econômicos é uma das ações de maior impacto que um indivíduo pode tomar para reduzir sua pegada financeira e ambiental. Contudo, um dos maiores obstáculos para a mudança de hábitos é a invisibilidade do consumo.

Atividades cotidianas, como o banho, são praticadas sem uma noção clara de seu impacto imediato. O usuário comum não sabe quantos litros de água ou quilowatts-hora está consumindo em tempo real, nem o custo financeiro associado a cada minuto adicional. A única forma de feedback é a fatura mensal, que chega semanas após o consumo. Este longo intervalo quebra o ciclo de causa e efeito, tornando impossível para o usuário associar uma ação específica a um custo direto, o que desincentiva a economia.

É neste cenário de "lacuna de informação" que a tecnologia da Internet das Coisas (IoT) surge como uma ferramenta de transformação. A IoT proporciona aos objetos do dia-a-dia a capacidade de coletar dados e interagir com o usuário. Esta conectividade, aliada a sistemas embarcados de baixo custo, viabiliza o desenvolvimento de "Casas Inteligentes" (Smart Homes), onde o gerenciamento de recursos pode ser otimizado, fornecendo dados antes inacessíveis.

O diferencial deste projeto, portanto, não reside apenas na coleta dos dados, mas na sua apresentação imediata ao usuário, fechando o ciclo de feedback. As informações coletadas são processadas localmente e disponibilizadas em tempo real.

A relevância deste trabalho se justifica por sua capacidade de transformar o ato abstrato de "economizar energia" em uma meta concreta e mensurável. Ao fornecer o conhecimento necessário para compreender o impacto financeiro e ambiental de uma ação rotineira, o sistema torna o invisível visível. Com isso, o projeto atua como uma ferramenta eficaz para a conservação de recursos, incentivando o consumo consciente e a redução de desperdícios, alinhando o interesse econômico do indivíduo com a necessidade global de sustentabilidade.