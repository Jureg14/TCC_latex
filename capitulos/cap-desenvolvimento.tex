
\chapter{Materiais e métodos}
%Deve conter tudo o que foi feito para construir o sistema.
Esta seção descreve o processo de desenvolvimento, construção e validação do dispositivo, dissertando sobre os materiais usados, sua montagem, programação e funcionamento. 

\section{Delineação de requerimentos}

Nesta etapa foram definidos os objetivos a serem alcançados e as estratégias para executá-los, permitindo, dessa forma uma visão mais clara do problema e, consequentemente, da solução do mesmo.

Com base em uma inspeção visual e experimental dos sistemas necessarios para a operação de um chuveiro elétrico, foram determinados e organizados os requerimentos no Quadro~\ref{quad:leituras_requeridas}

%Tabela requerimento sensores
\begin{tabframed}[htb]%% Ambiente tabframed
%\captionsetup{width=0.5\textwidth}%% Largura da legenda
\caption{Leituras requeridas}%% Legenda
\label{quad:leituras_requeridas}%% Rótulo
\renewcommand{\arraystretch}{1.5}
\begin{tabular}{|p{12cm}|}
\cline{1-1} 
%\multicolumn{1}{|c|}{\textbf{\centering Descrição das Funcionalidades}}\\ \cline{1-1} 
 Leitura do fluxo de água \\ \cline{1-1}
 Leitura da potência consumida \\ \cline{1-1}
\end{tabular}
\fonte{Autoria própria (2025)}%% Fonte
\addcontentsline{loge}{tabframed}{\protect\numberline{\thetabframed}Leituras requeridas}
\end{tabframed}

 Considerando que a tensão da rede elétrica residencial é geralmente estável e que o chuveiro elétrico comporta-se como uma carga puramente resistiva, foi decidido tratar a tensão elétrica como uma constante no cálculo da potência. 
 
 Dessa forma o sensor de tensão foi substituido por uma chave seletora com a qual o usuario deve determinar a tensão nominal do chuveiro (por exemplo, 127 V ou 220 V). Isso permite simplificar significativamente a arquitetura do sistema e reduzir seus custos de implementação, mas ainda assim, mantém a precisão das medições dentro de limites aceitáveis para a aplicação proposta.

 Assim, podemos organizar no Quadro~\ref{quad:funcionalidades_requeridas} a lista completa de requerimentos.

%Tabela requerimento software
\begin{tabframed}[htb]%% Ambiente tabframed
%\captionsetup{width=0.5\textwidth}%% Largura da legenda
\caption{Requerimentos}%% Legenda
\label{quad:funcionalidades_requeridas}%% Rótulo
\renewcommand{\arraystretch}{1.5}
\begin{tabular}{|p{12cm}|}
\cline{1-1} 
%\multicolumn{1}{|c|}{\textbf{\centering Descrição das Funcionalidades}}\\ \cline{1-1} 
 Coletar dados dos sensores \\ \cline{1-1}
 Calculo de potência \\ \cline{1-1}
 Calculo do fluxo de água \\ \cline{1-1}
 Gerenciamento das informações coletadas \\ \cline{1-1}
 Conexão a rede WiFi \\ \cline{1-1}
 Hosteamento de um webserver \\ \cline{1-1}
 Feedback instantâneo via LCD \\ \cline{1-1}
 Interface clara e funcional do WebApp \\ \cline{1-1}
\end{tabular}
\fonte{Autoria própria (2025)}%% Fonte
\addcontentsline{loge}{tabframed}{\protect\numberline{\thetabframed}Requerimentos}
\end{tabframed}

A partir desse levantamento de dados, foi possivel determinar o escopo do software e hardware para o projeto, os quais são explicados de maneira mais rica nos capitulos a seguir.


\section{Materiais utilizados}

%Tabelar as alternativas, explicando quais foram escolhidas e porque

Para a montagem do projeto foram usados componentes de facil acesso, os quais estão listados na Tabela~\ref{tab:listaDeComponentes}

\begin{table}[!htb]
 % Luiz - O texto do caption da tabela/quadro deve ser do tamanho da tabela, então utilize a linha a seguir para conseguir esse efeito
 \captionsetup{width=0.83\textwidth}
 \centering
 \caption{\label{tab:listaDeComponentes}Lista de componentes usados}
 \begin{tabular}{p{6cm} cccc}
 	\hline

            \multicolumn{1}{c}{Componente}     & Quantidade \\ \hline
            Fonte 5V USB	               & 1	      \\
            ESP32 espressif	               & 1	      \\
            YF-S201 	                   & 1	      \\
            SCT-013 100A	               & 1	      \\
            LCD ST7735	                   & 1	      \\
            Chave Tactil                   & 1	      \\
            Chave switch                   & 1	      \\
            Conector p3                    & 1	      \\
            Capacitor 10$\mu$ 50V          & 1	      \\
            Resistor 22$\Omega$            & 1	      \\
            Resistor 10K$\Omega$           & 3	      \\
            \hline
 \end{tabular}
 \fonte{Autoria própria (2025)}
 \end{table}

O  ESP32 Espressif foi escolhido pois atende as necessidades do projeto, atendendo as requisitos como numero de pinos I/O, quantidade de armazenamento e comunicação sem fio WiFi. Alem disso pode ser obtido a um baixo custo, e apresenta uma boa documentação.

O LCD em uso é o ST7735, é usado em conjunto com um botão para servir de interface para o usuário, foi escolhido porque, mesmo dado seu baixo custo, consiste em uma tela pequena, de alta resolução e colorida, o que resulta em boa plataforma para se desenvolver uma boa experiência de uso para o usuário.

O sensor de corrente é o transformador SCT-013 de 100 A, ele apresenta um bom custo-benefício, permitindo a leitura da corrente consumida pelo chuveiro de maneira não invasiva. 
O modelo escolhido possui duas particularidades, as quais devem ser levadas em conta durante o design:
\begin{itemize}
    \item A primeira é a presença um conector p3 macho, o que levou a inclusão de um conector p3 femea no circuito;
    \item A segunda é que diferente das versões que geram uma tensão alternada proporcional a corrente sendo medida, que essa variação do sensor gera uma corrente alternada, essa corrente precisa processada por meio de um circuito de condicionamento, o qual atua transformando a corrente alternada em uma tensão que pode ser lida e processada pelo microcontrolador.
\end{itemize}
%O sensor YF-S201 é acessível e, por meio de experimentos, demonstrou operar adequadamente com 3,3 V
O sensor de vazão de água YF-S201, é acessivel a um preço baixo, e, por meio de experimentos, demonstrou operar adequadamente com 3,3 V, o que é mais baixo que sua tensão nominal de operação de 5 V. Isso é relevante pois simplifica seu uso com o microcontrolador escolhido, o ESP32, cujos pinos suportam somente tensões até 3.6 V.

Os conectores foram usados por conveniencia, mas todos os sensores podem ser soldados diretamente na placa, o que pode vir a gerar menos problemas ao longo do tempo devido a natureza umida do ambiente onde o sistema é usado.

\section{Montagem}
%o que foi montado, porque e como.
Antes de se iniciar a montagem em si, foi elaborado o diagrama elétrico dos componentes no software KiCad. Esse diagrama esta ilustrado na figura~\ref{fig:kicad_circuit}.

\begin{figure}[!htb]%% Ambiente figure
     %\captionsetup{width=0.55\textwidth}%% Largura da legenda
     \caption{Esquematica}%% Legenda
     \label{fig:kicad_circuit}%% Rótulo
     \includegraphics[scale=0.5]{figuras/kicad_circuit.png}%% Dimensões e localização
     \fonte{Autoria própria (2025)}%% Fonte
     \addcontentsline{loge}{figure}{\protect\numberline{\thefigure}Esquematica}
 \end{figure}

Como ilustrado na Figura~\ref{fig:sensore_de_fluxo}, o sensor de fluxo de água e a luva foram instalados entre o cano e o chuveiro. Isso faz com que toda a água que vai para o chuveiro passe pelo sensor e seja medida pelo sistema.

\begin{figure}[htbp]%% Ambiente figure
     %\captionsetup{width=0.55\textwidth}%% Largura da legenda
     \caption{Sensor de fluxo instalado}%% Legenda
     \label{fig:sensore_de_fluxo}%% Rótulo
     \includegraphics[scale=0.1]{figuras/sensore_de_fluxo.jpg}%% Dimensões e localização
     \fonte{Autoria própria (2025)}%% Fonte
     \addcontentsline{loge}{figure}{\protect\numberline{\thefigure}Sensor de fluxo instalado}
 \end{figure}

Com devido cuidado, o transformador de corrente pode ser simplesmente preso no fio, como na Figura~\ref{fig:TC_montado}. 

\begin{figure}[htbp]%% Ambiente figure
     %\captionsetup{width=0.55\textwidth}%% Largura da legenda
     \caption{transformador de corrente posicionado}%% Legenda
     \label{fig:TC_montado}%% Rótulo
     \includegraphics[scale=0.1]{figuras/TC_montado.jpg}%% Dimensões e localização
     \fonte{Autoria própria (2025)}%% Fonte
     \addcontentsline{loge}{figure}{\protect\numberline{\thefigure}transformador de corrente posicionado}
 \end{figure}

A caixa para o circuito elétrico, para o LCD, e o botão podem ser fixados na parededo banheiro proximo ao chuveiro como na Figura~\ref{fig:Sistema_montado}. 

\begin{figure}[!htb]%% Ambiente figure
     %\captionsetup{width=0.55\textwidth}%% Largura da legenda
     \caption{Sistema montado e implementado}%% Legenda
     \label{fig:Sistema_montado}%% Rótulo
     \includegraphics[scale=0.1, angle=-90]{figuras/Sistema_montado.jpg}%% Dimensões e localização
     \fonte{Autoria própria (2025)}%% Fonte
     \addcontentsline{loge}{figure}{\protect\numberline{\thefigure}Sistema montado e implementado}
 \end{figure}

A caixa do circuito deve ser instalada o mais alto possivel, já a caixa do LCD e o botão devem ser fixados em um relativamente ponto alto, de forma a minimizar o contato com a água, mas não tão alto a ponto de serem inacessiveis pelo usuario.

\section{Desenvolvimento do software}

Com o hardware consolidado, o desenvolvimento do software, além da funcionalidade, deve priorizar a experiencia do usuario, gerando uma interface amigavel e intuitiva. A implementação tecnica foi executada na IDE Visual Studio Code, atraves do plugin PlataformIO, e ocorreu em duas frentes, firmware e o aplicativo web.
%O firmware foi escrito em C++, lida com a leitura de dados, do lcd, da conexão com a internet e do gerenciamento de um simples sistema de arquivos implementado dentro do microcontrolador com ajuda da livraria LittleFS. Graças a esse sistema de arquivo, a base de dados dos banhos é organizada na forma de um arquivo csv, e o  aplicativo web pode ser construido de maneira tradicional, com HTML, CSS e JavaScript.

\subsection{Firmware}

O firmware foi escrito em C++, é responsavel por toda a lógica de detecção de banhos, leitura, gerenciamento e persistência dos dados. Seu funcionamento é ilustrado na Figura~\ref{fig:firmware_flowchart}.

\begin{figure}[htbp]%% Ambiente figure
     %\captionsetup{width=0.55\textwidth}%% Largura da legenda
     \caption{Fluxograma do Firmware}%% Legenda
     \label{fig:firmware_flowchart}%% Rótulo
     \includegraphics[scale=0.18]{figuras/firmware_flowchart.png}%% Dimensões e localização
     \fonte{Autoria própria (2025)}%% Fonte
     \addcontentsline{loge}{figure}{\protect\numberline{\thefigure}Fluxograma do Firmware}
 \end{figure}
\FloatBarrier
\subsection{Web App}
 A solução web app foi escolhido porque, ao contrário de um aplicativo instalado, ele pode ser acessado em qualquer dispositivo com suporte a um navegador moderno, sem a necessidade de instalar softwares adicionais. Isso torna o acesso mais fácil para os usuários e simplifica o processo de desenvolvimento.
 
 Foi desenvolvido como metodo primario de interagir com o projeto, fornecendo tanto um feedback em tempo real, quanto o acesso aos dados coletados ao longo do tempo. A Figura~\ref{fig:Web_Mockup} ilustra o protótipo de interface que orientou o processo de desenvolvimento.

 \begin{figure}[htbp]%% Ambiente figure
     %\captionsetup{width=0.55\textwidth}%% Largura da legenda
     \caption{Protótipo de interface}%% Legenda
     \label{fig:Web_Mockup}%% Rótulo
     \includegraphics[scale=0.2]{figuras/Web_Mockup.jpg}%% Dimensões e localização
     \fonte{Autoria própria (2025)}%% Fonte
     \addcontentsline{loge}{figure}{\protect\numberline{\thefigure}Protótipo de interface}
 \end{figure}

A interface web foi implementada com tecnologias web padrão, HTML, CSS e JavaScript, de forma semelhante a um site tradicional. O uso de JavaScript permite o processamento dos dados recebidos diretamente no navegador, aliviando a carga computacional do microcontrolador, que já é responsável pela análise contínua dos sensores.

%A interface web foi desenvolvida utilizando tecnologias web padrão, HTML, CSS e JavaScript, seguindo uma abordagem semelhante à de um site convencional.

%O software desenvolvido gera uma interface web, acessivel por meio do navegador de um celular ou computador. Essa interface é o método mais completo de se interagir com o sistema, ela é dividida em duas partes principais, o painel, ilustrado na Figura~\ref{fig:WebUi_1}, e o historico, ilustrado na Figura~\ref{fig:WebUi_2} e Figura~\ref{fig:WebUi_3}.

 No painel, são mostradas as informações em tempo real sobre o consumo de água e energia elétrica do banho, assim como os custos do banho anterior.
 No final da pagina temos dois botões, o da esquerda força a atualização da pagina, e o da direita abra a aba de configuração.

\begin{figure}[htbp]%% Ambiente figure
     %\captionsetup{width=0.55\textwidth}%% Largura da legenda
     \caption{Painel}%% Legenda
     \label{fig:WebUi_1}%% Rótulo
     \includegraphics[scale=0.7]{figuras/WebUi_1.png}%% Dimensões e localização
     \fonte{Autoria própria (2025)}%% Fonte
     \addcontentsline{loge}{figure}{\protect\numberline{\thefigure}Painel}
 \end{figure}

Na aba de configuração, o usuario pode definir variaveis como o custo de energia, custo da água, e o reset automatico do sistema, mostrado na figura~\ref{fig:WebUi_cfg}.

\begin{figure}[htbp]%% Ambiente figure
     %\captionsetup{width=0.55\textwidth}%% Largura da legenda
     \caption{Página configuração}%% Legenda
     \label{fig:WebUi_cfg}%% Rótulo
     \includegraphics[scale=0.7]{figuras/WebUi_cfg.png}%% Dimensões e localização
     \fonte{Autoria própria (2025)}%% Fonte
     \addcontentsline{loge}{figure}{\protect\numberline{\thefigure}Página configuração}
 \end{figure}

 No historico, ilustrado pelas Figura~\ref{fig:WebUi_2} e Figura~\ref{fig:WebUi_3}, temos tres graficos de barras, ilustrando os custos dos banhos. 
 Em ordem, de cima para baixo:
 \begin{itemize}
     \item No topo temos o grafico com os custos dos banhos do dia.
     \item No meio temos o grafico com os custos de todos os banhos das ultimas duas semanas.
     \item O terceiro e ultimo grafico mostra o custo medio de um banho nos ultimos meses.
 \end{itemize}

 \begin{figure}[htbp]%% Ambiente figure
     %\captionsetup{width=0.55\textwidth}%% Largura da legenda
     \caption{Historico parte superior}%% Legenda
     \label{fig:WebUi_2}%% Rótulo
     \includegraphics[scale=0.7]{figuras/WebUi_2.png}%% Dimensões e localização
     \fonte{Autoria própria (2025)}%% Fonte
     \addcontentsline{loge}{figure}{\protect\numberline{\thefigure}Historico parte superior}
 \end{figure}

No final da pagina temos uma aba que mostra, de maneira menos ilustrada, porem mais detalhada, uma tabela com todos os banhos tomados que estão guardados na memoria do dispositivo. Aqui essa tabela pode sei baixada na forma de um arquivo csv.

 \begin{figure}[htbp]%% Ambiente figure
     %\captionsetup{width=0.55\textwidth}%% Largura da legenda
     \caption{Histórico parte inferior}%% Legenda
     \label{fig:WebUi_3}%% Rótulo
     \includegraphics[scale=0.7]{figuras/WebUi_3.png}%% Dimensões e localização
     \fonte{Autoria própria (2025)}%% Fonte
     \addcontentsline{loge}{figure}{\protect\numberline{\thefigure}Histórico parte inferior}
 \end{figure}
\FloatBarrier
Dessa forma, além das ferramentas fornecidas pela interface web, o usuario pode, através do arquivo csv fornecido, usar outras ferramentas para melhor analizar seus habitos e mais facilmente identificar padrões em seu consumo.

%falar do LCD
\subsection{Interface LCD}

Embora a interface web ofereça flexibilidade no acesso e pós processamento de dados, o feedback em tempo real exigiria que o usuário consultasse um dispositivo móvel, como um smartphone, durante o banho, o que além de ser pouco prático, é muitas vezes inviável, especialmente considerando que a maioria desses dispositivos não é a prova d'água. Dessa forma, optou-se pelo uso de uma LCD com  interface simplificada como forma de fornecer feedback em tempo real, diretamente no ambiente do banheiro, para o usuario.

A interface LCD é composta por duas telas: a primeira fornece o consumo em tempo real durante o banho (Figura~\ref{fig:LCD_telaBanho}), a segunda mostra um relatório pós banho, comparando os custos do banho atual com a média dos custos dos banhos anteriores (Figura~\ref{fig:LCD_telaPósBanho}).
%A interface LCD é composta por duas telas distintas. A primeira exibe, em tempo real durante o banho, as informações de consumo de água, energia e custo acumulado (Figura~XX). A segunda tela apresenta, ao final do banho, um relatório comparativo que mostra o custo do banho atual em relação à média dos banhos anteriores (Figura~XXX), oferecendo ao usuário um feedback imediato e contextualizado sobre seu desempenho.

 \begin{figure}[htbp]%% Ambiente figure
     %\captionsetup{width=0.55\textwidth}%% Largura da legenda
     \caption{Feedback via LCD}%% Legenda
     \label{fig:LCD_telaBanho}%% Rótulo
     \includegraphics[scale=0.1]{figuras/LCD_telaBanho.jpg}%% Dimensões e localização
     \fonte{Autoria própria (2025)}%% Fonte
     \addcontentsline{loge}{figure}{\protect\numberline{\thefigure}Feedback via LCD}
 \end{figure}



 \begin{figure}[htbp]%% Ambiente figure
     %\captionsetup{width=0.55\textwidth}%% Largura da legenda
     \caption{Relatório pós banho}%% Legenda
     \label{fig:LCD_telaPósBanho}%% Rótulo
     \includegraphics[scale=0.1]{figuras/LCD_telaPósBanho.jpg}%% Dimensões e localização
     \fonte{Autoria própria (2025)}%% Fonte
     \addcontentsline{loge}{figure}{\protect\numberline{\thefigure}Relatório pós banho}
 \end{figure}

%Embora a interface web ofereça flexibilidade no acesso aos dados, o feedback em tempo real exigiria que o usuário consultasse um dispositivo móvel — como um smartphone — durante o banho, o que é pouco prático e muitas vezes inviável, especialmente considerando que a maioria desses dispositivos não é à prova d’água. Diante dessa limitação, optou-se pela integração de um display LCD com interface simplificada, capaz de fornecer ao usuário informações imediatas sobre o consumo diretamente no ambiente do banheiro, de forma segura e acessível.

\chapter{Resultados e discussões}
%Resultados PRATICOS
Os primeiros resultados obtidos por meio da montagem e operação do projeto são apresentados nessa seção. Aqui se tem por objetivo a visualização e validação dos dados coletados, assim como os resultados dos mesmos.

\section{Validação Experimental}
%desenrolar sobre como as leituras foram validadas
Apresenta-se a seguir o passo a passo do uso de ferramentas para validação das leituras obtidas pelo sistema, garantindo seu correto funcionamento.

\subsection{Ferramentas utilizadas}

Para medir a vazão real do chuveiro, foram usados um balde, uma balança digital (ilustrados na Figura~\ref{fig:balança&balde}) e um cronômetro. O método consistiu em coletar água do chuveiro no balde durante um intervalo de tempo cronometrado, e a massa de água coletada pelo balde foi registrada pela balança.

%O procedimento consistiu em coletar a água do chuveiro no balde durante um intervalo de tempo previamente medido com o cronômetro, enquanto a massa da água coletada foi registrada pela balança.

 \begin{figure}[htbp]%% Ambiente figure
     %\captionsetup{width=0.55\textwidth}%% Largura da legenda
     \caption{Balança digital e balde}%% Legenda
     \label{fig:balança&balde}%% Rótulo
     \includegraphics[scale=0.1]{figuras/balança&balde.jpg}%% Dimensões e localização
     \fonte{Autoria própria (2025)}%% Fonte
     \addcontentsline{loge}{figure}{\protect\numberline{\thefigure}Balança digital e balde}
 \end{figure}

Alicate amperimetro na Figura~\ref{fig:amperimetro}.

 \begin{figure}[htbp]%% Ambiente figure
     %\captionsetup{width=0.55\textwidth}%% Largura da legenda
     \caption{Alicate amperimetro}%% Legenda
     \label{fig:amperimetro}%% Rótulo
     \includegraphics[scale=0.1]{figuras/amperimetro.jpg}%% Dimensões e localização
     \fonte{Autoria própria (2025)}%% Fonte
     \addcontentsline{loge}{figure}{\protect\numberline{\thefigure}Alicate amperimetro}
 \end{figure}

\subsection{Validação dos valores medidos}

A partir dos dados coletados de diversos testes realizados, duas tabelas foram organizadas: a Tabela~\ref{tab:tabCorrente}, com as medições de corrente elétrica, e a Tabela~\ref{tab:tabVazao}, com os valores de vazão de água. Em ambas, a primeira coluna contém os valores registradas pelo sistema desenvolvido neste trabalho; a segunda coluna tem os valores de referência medidos manualmente com as ferramentas de medição; a terceira coluna apresenta o erro absoluto entre as duas leituras, utilizado para avaliar a precisão do sistema.

%A partir dos dados obtidos nos diversos testes realizados, foram organizadas duas tabelas: a Tabela XX, com as medições de corrente elétrica, e a Tabela XXX, com os valores de vazão de água. Em ambas, a primeira coluna apresenta as leituras registradas pelo sistema desenvolvido neste trabalho; a segunda coluna, os valores de referência obtidos por meio de instrumentos de medição calibrados (medição manual); e a terceira coluna, o erro absoluto entre as duas leituras, utilizado para avaliar a precisão do protótipo.

\begin{table}[!htb]
 % Luiz - O texto do caption da tabela/quadro deve ser do tamanho da tabela, então utilize a linha a seguir para conseguir esse efeito
 \captionsetup{width=0.83\textwidth}
 \centering
 \caption{\label{tab:tabCorrente}Leitura de corrente}
 \begin{tabular}{p{6cm} cccc}
 	          \hline
            \multicolumn{1}{c}{leituras registradas (A)}     & Valores de referência (A) & Erro absoluto (\%) \\ 
            \hline
            40,1	                                         & 40,2	                     & 0,25    \\
            24,4	                                         & 24,3	                     & 0,41    \\ 
            40	                                             & 39,7	                     & 0,76    \\
            24,2	                                         & 24,5	                     & 1,22    \\
            24,1	                                         & 24,7	                     & 2,43    \\
            \hline
 \end{tabular}
 \fonte{Autoria própria (2025)}
 \end{table}

Como se pode ver, os valores de corrente registrados pelo sistema são bem próximos dos valores medidos, com o maior erro sendo menor que 3\%. 

 Antes de serem tabelados, os valores de vazão foram calculados conforme demonstrado na Tabela~\ref{tab:calculoVazao}. Nota-se que foi consideranda a densidade da água aproximadamente igual a 1 g/mL, e a massa de água coletada foi diretamente convertida em volume, de modo que 1 grama corresponde a 1 mililitro.

 \begin{table}[!htb]
 % Luiz - O texto do caption da tabela/quadro deve ser do tamanho da tabela, então utilize a linha a seguir para conseguir esse efeito
 \captionsetup{width=0.83\textwidth}
 \centering
 \caption{\label{tab:calculoVazao}Vazão calculada}
 \begin{tabular}{p{6cm} cccc}
 	          \hline
            \multicolumn{1}{c}{Volume de água (L)}     & tempo (segundos) & Vazão calculada (L/min) \\ 
            \hline
            1,825	                                       & 50,1	              & 2,18    \\
            1,764	                                       & 31,34	              & 3,37    \\
            2,135	                                       & 43,5	              & 2,95    \\
            1,957	                                       & 45,22	              & 2,59    \\
            1,591	                                       & 53,37	              & 1,79    \\
            1,307	                                       & 72,44	              & 1,08    \\
            \hline
 \end{tabular}
 \fonte{Autoria própria (2025)}
 \end{table}

Os valores de vazão calculados na Tabela~\ref{tab:calculoVazao} foram utilizados como referência na Tabela~\ref{tab:tabVazao}, permitindo uma comparação direta com as medições registradas pelo sistema.

 \begin{table}[!htb]
 % Luiz - O texto do caption da tabela/quadro deve ser do tamanho da tabela, então utilize a linha a seguir para conseguir esse efeito
 \captionsetup{width=0.83\textwidth}
 \centering
 \caption{\label{tab:tabVazao}Leitura de vazão}
 \begin{tabular}{p{6cm} cccc}
 	          \hline
            \multicolumn{1}{c}{leituras registradas (L/min)}     & Valores de referência (L/min) & Erro absoluto (\%) \\ 
            \hline
            2,6	                                                 & 2,59	                         & 0,38    \\
            3,4	                                                 & 3,37	                         & 0,89    \\
            3,0	                                                 & 2,95	                         & 1,69    \\
            1,9	                                                 & 2,18	                         & 12,84   \\
            1,2	                                                 & 1,79	                         & 32,96   \\ 
            0,5	                                                 & 1,08	                         & 53,70   \\
            \hline
 \end{tabular}
 \fonte{Autoria própria (2025)}
 \end{table}

É notavel, pelos valores na terceira coluna da Tabela~\ref{tab:tabVazao}, que a performance do sensor de vazão degrada significativamente em condições de baixo fluxo de água.

Por fim, o desempenho geral do sistema mostrou-se aceitável para a aplicação proposta. O sensor de corrente demonstrou consistência ao longo das medições, com desvios máximos de 2,53\% em relação ao valor esperado. Já o medidor de vazão se demonstrou incapaz de obter leituras de vazão baixas sem um erro significativo, mas, deve se considerar que no chuveiro utilizado, uma vazão confortavel para o banho é acima de 2,5 L/min , o que está dentro da faixa de operação com erro aceitavel. Assim, dentro da faixa de uso típico que se espera do sistema, o mesmo cumpre sua função satisfatóriamente.
%Por fim, o desempenho geral do sistema mostrou-se aceitável para a aplicação proposta. O sensor de corrente demonstrou consistência ao longo das medições, com desvios máximos de 2,53% em relação ao valor esperado. Já o medidor de vazão, embora impreciso em faixas muito baixas, opera com erro tolerável acima de 2,5 L/min — valor que corresponde à vazão mínima considerada confortável para banho no chuveiro utilizado. Assim, dentro da faixa de uso típico, o sistema cumpre adequadamente sua função.

\section{Análise dos Dados Coletados pelo Sistema}

O sistema foi monitorado por duas semanas, como ilustrada pelo grafico na Figura~\ref{fig:WebPrimeirasDuasSemanas}. Nos primeiros dias, eram gastos, por dia, entorno de R\$7 reais com banhos. Após o usuário se acostumar com o feedback em tempo real, houve uma tendencia de queda dos gastos, que cairam para R\$5 por dia, demonstrando a eficácia do sistema em promover a conscientização.

 \begin{figure}[htbp]%% Ambiente figure
     %\captionsetup{width=0.55\textwidth}%% Largura da legenda
     \caption{Gráfico de custo diário ao longo de duas semanas}%% Legenda
     \label{fig:WebPrimeirasDuasSemanas}%% Rótulo
     \includegraphics[scale=0.70]{figuras/WebPrimeirasDuasSemanas.jpg}%% Dimensões e localização
     \fonte{Autoria própria (2025)}%% Fonte
     \addcontentsline{loge}{figure}{\protect\numberline{\thefigure}Gráfico de custo diário ao longo de duas semanas}
 \end{figure}

\section{Limitações técnicas}
%falar que a potência é estimada por causa da falta de leitura da  tensão
A potência elétrica calculada se trata de uma estimativa, pois a tensão é assumida como constante, simplificação adotada de forma a simplificar o projeto, reduzindo seu custo e complexidade.

O sensor de corrente apresentou pequenas variações no valor medido em função de seu posicionamento no fio, o que pode afetar levemente a precisão dos resultados

O sensor de vazão de água adotado demonstrou uma degradação significativa em sua performance em condições de baixo fluxo, o que impossibilita o uso em duchas de baixa vazão.
%a performance do sensor de vazão degrada significativamente em condições de baixo fluxo