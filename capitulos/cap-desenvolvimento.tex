
\chapter{Materiais e métodos}
%Deve conter tudo o que foi feito para construir o sistema.
Esta seção descreve o processo de desenvolvimento, construção e validação do dispositivo, dissertando sobre os materiais usados, sua montagem, programação e funcionamento. 

\section{Delineação de requerimentos}

Nesta etapa foram definidos os objetivos a serem alcançados e as estratégias para executá-los, permitindo, dessa forma, uma visão mais clara do problema e, consequentemente, da solução do mesmo.
Com base em uma inspeção visual e experimental dos sistemas necessários para a operação de um chuveiro elétrico, foram determinados e organizados os requerimentos no Quadro~\ref{quad:leituras_requeridas}.

%Tabela requerimento sensores
\begin{tabframed}[htb]%% Ambiente tabframed
%\captionsetup{width=0.5\textwidth}%% Largura da legenda
\caption{Leituras requeridas}%% Legenda
\label{quad:leituras_requeridas}%% Rótulo
\renewcommand{\arraystretch}{1.5}
\begin{tabular}{|p{12cm}|}
\cline{1-1} 
%\multicolumn{1}{|c|}{\textbf{\centering Descrição das Funcionalidades}}\\ \cline{1-1} 
 Leitura do fluxo de água \\ \cline{1-1}
 Leitura da potência consumida \\ \cline{1-1}
\end{tabular}
\fonte{Autoria própria (2025)}%% Fonte
%\addcontentsline{loge}{tabframed}{\protect\numberline{\thetabframed}Leituras requeridas}
\end{tabframed}

Diante do comportamento predominantemente resistivo do chuveiro elétrico, e considerando que, de acordo com a \citeonline{ANEEL:PRODISTMod8_v11} as flutuações típicas da tensão de fornecimento em redes residenciais devem permanecer dentro da faixa de +5\% e -8\% do valor nominal, optou-se, como simplificação de projeto, por assumir a tensão nominal como constante. Essa abordagem permite eliminar o sensor de tensão, substituindo-o por uma chave seletora que permite ao usuário informar a tensão nominal de operação do chuveiro (por exemplo, 127 V ou 220 V). 

A decisão está alinhada à natureza da aplicação, voltada à conscientização do consumo e não ao faturamento, onde pequenos erros de medição são toleráveis em favor de ganhos em robustez, simplicidade e redução de custos. Os requisitos finais do sistema são consolidados no Quadro~\ref{quad:funcionalidades_requeridas}.


%Tabela requerimento software
\begin{tabframed}[htb]%% Ambiente tabframed
%\captionsetup{width=0.5\textwidth}%% Largura da legenda
\caption{Requerimentos}%% Legenda
\label{quad:funcionalidades_requeridas}%% Rótulo
\renewcommand{\arraystretch}{1.5}
\begin{tabular}{|p{12cm}|}
\cline{1-1} 
%\multicolumn{1}{|c|}{\textbf{\centering Descrição das Funcionalidades}}\\ \cline{1-1} 
 Coletar dados dos sensores \\ \cline{1-1}
 Calcular de potência \\ \cline{1-1}
 Calcular o fluxo de água \\ \cline{1-1}
 Gerenciar as informações coletadas \\ \cline{1-1}
 Conectar a rede WiFi \\ \cline{1-1}
 Hostear uma aplicação \textit{web} \\ \cline{1-1}
 Apresentar leituras instantaneamente por meio de uma tela \\ \cline{1-1}
 Gerenciar uma interface clara e funcional na aplicação \textit{web} \\ \cline{1-1}
\end{tabular}
\fonte{Autoria própria (2025)}%% Fonte
%\addcontentsline{loge}{tabframed}{\protect\numberline{\thetabframed}Requerimentos}
\end{tabframed}

A partir desse levantamento de dados, foi possível determinar o escopo do \textit{software} e \textit{hardware} para o projeto, os quais são explicados de maneira mais descritiva nos capítulos a seguir.


\section{Materiais utilizados}

%Tabelar as alternativas, explicando quais foram escolhidas e porque

Para a montagem do projeto foram usados componentes cujo custo-benefício foi atrativo, os quais estão listados na Tabela~\ref{tab:listaDeComponentes}

\begin{table}[!htb]
 % Luiz - O texto do caption da tabela/quadro deve ser do tamanho da tabela, então utilize a linha a seguir para conseguir esse efeito
 \captionsetup{width=0.83\textwidth}
 \centering
 \caption{\label{tab:listaDeComponentes}Lista de componentes usados}
 \begin{tabular}{p{6cm} cccc}
 	\hline

            \multicolumn{1}{c}{Componente}     & Quantidade \\ \hline
            Fonte 5V USB	               & 1	      \\
            ESP32 espressif	               & 1	      \\
            YF-S201 	                   & 1	      \\
            SCT-013-000	               & 1	      \\
            LCD ST7735	                   & 1	      \\
            Chave Táctil                   & 1	      \\
            Chave switch                   & 1	      \\
            Conector p3                    & 1	      \\
            Capacitor 10$\mu$ 50V          & 1	      \\
            Resistor 22$\Omega$            & 1	      \\
            Resistor 10K$\Omega$           & 3	      \\
            \hline
 \end{tabular}
 \fonte{Autoria própria (2025)}
 \end{table}

O microcontrolador ESP32, desenvolvido pela Espressif, foi selecionado como plataforma central do sistema por atender aos requisitos do projeto: disponibilidade suficiente de pinos de entrada e saída, capacidade de armazenamento e suporte nativo à comunicação sem fio via WiFi \cite{ESPRESSIF:ESP32DS}. Além de apresentar  ampla documentação técnica, o que facilita o desenvolvimento e a manutenção.

Para a interface com o usuário, optou-se pelo \textit{display} gráfico ST7735, um módulo com uma tela de cristal líquido (LCD, do inglês \textit{Liquid Crystal Display}) colorido de 1,8 polegadas, com resolução de 160 × 128 pixels \cite{LCDWiki:MSP1803}. O dispositivo oferece boa qualidade visual e densidade de informação, sendo adequado para a implementação de uma interface gráfica intuitiva e eficaz.

O monitoramento da corrente elétrica foi realizado com o sensor de efeito indutivo SCT-013-000, um transformador de corrente  não invasivo que suporta leituras de até 100 A \cite{MAKERHERO:SCT013:2015}. O modelo empregado apresenta duas características relevantes para o projeto: 
\begin{itemize}
    \item Conector físico: possui saída tipo plugue P3 macho, exigindo a incorporação de um conector P3 fêmea na placa de interface;
    \item Saída em corrente: diferentemente de versões que entregam tensão proporcional à corrente primária, este sensor fornece uma corrente alternada secundária proporcional à corrente primária, cuja conversão em tensão utilizável demanda um circuito de condicionamento, ilustrado pela Figura~\ref{fig:kicad_circuit}, composto por carga resistiva (\textit{burden resistor}) e, opcionalmente, etapas de filtragem e amplificação, de forma a viabilizar a leitura pelo conversor analógico-digital (ADC) do ESP32.
\end{itemize}

O sensor de vazão YF-S201 apresenta rosca padrão de meia polegada, compatível com os diâmetros usuais das conexões hidráulicas de chuveiros elétricos, o que permite sua instalação com o mínimo de adaptações mecânicas, necessitando apenas de uma luva roscável de meia polegada, como ilustrado na Figura~\ref{fig:sensore_de_fluxo}.

%O sensor de vazão de água YF-S201 utiliza da mesma rosca de 1/2 polegada utilizada pelo encanamento do chuveiro, o que facilita sua instalação, por meio de experimentos, demonstrou operar adequadamente com 3,3 V, o que é mais baixo que sua tensão nominal de operação de 5 V. Isso é relevante, pois simplifica seu uso com o microcontrolador escolhido, o ESP32, cujos pinos suportam somente tensões até 3,6 V.

Os conectores foram usados por conveniência, mas todos os sensores podem ser soldados diretamente na placa, o que pode vir a gerar menos problemas ao longo do tempo devido à natureza úmida do ambiente onde o sistema é usado.

\section{Montagem}
%o que foi montado, porque e como.
Antes de se iniciar a montagem em si, foi elaborado o diagrama elétrico dos componentes no software KiCad. Esse diagrama está ilustrado na Figura~\ref{fig:kicad_circuit}.

\begin{figure}[!htb]%% Ambiente figure
     %\captionsetup{width=0.55\textwidth}%% Largura da legenda
     \caption{Esquemática elétrica do sistema}%% Legenda
     \label{fig:kicad_circuit}%% Rótulo
     \includegraphics[scale=0.5]{figuras/kicad_circuit.png}%% Dimensões e localização
     \fonte{Autoria própria (2025)}%% Fonte
     \addcontentsline{loge}{figure}{\protect\numberline{\thefigure}Esquemática}
 \end{figure}

Como ilustrado na Figura~\ref{fig:sensore_de_fluxo}, o sensor de fluxo de água e a luva roscável foram instalados entre o cano e o chuveiro. Isso faz com que toda a água que vai para o chuveiro passe pelo sensor e seja medida pelo sistema.

\begin{figure}[htbp]%% Ambiente figure
     %\captionsetup{width=0.55\textwidth}%% Largura da legenda
     \caption{Sensor de fluxo instalado}%% Legenda
     \label{fig:sensore_de_fluxo}%% Rótulo
     \includegraphics[scale=0.08]{figuras/sensore_de_fluxo.jpg}%% Dimensões e localização
     \fonte{Autoria própria (2025)}%% Fonte
     \addcontentsline{loge}{figure}{\protect\numberline{\thefigure}Sensor de fluxo instalado}
 \end{figure}

Com devido cuidado, o transformador de corrente pode ser simplesmente preso no fio, como na Figura~\ref{fig:TC_montado}. 

\begin{figure}[htbp]%% Ambiente figure
     %\captionsetup{width=0.55\textwidth}%% Largura da legenda
     \caption{Transformador de corrente posicionado}%% Legenda
     \label{fig:TC_montado}%% Rótulo
     \includegraphics[scale=0.12]{figuras/TC_montado.jpg}%% Dimensões e localização
     \fonte{Autoria própria (2025)}%% Fonte
     \addcontentsline{loge}{figure}{\protect\numberline{\thefigure}Transformador de corrente posicionado}
 \end{figure}

A caixa para o circuito elétrico, para o LCD  ser fixada na parede do banheiro próximo ao chuveiro como na Figura~\ref{fig:Sistema_montado}. 

\begin{figure}[!htb]%% Ambiente figure
     %\captionsetup{width=0.55\textwidth}%% Largura da legenda
     \caption{Sistema montado e implementado}%% Legenda
     \label{fig:Sistema_montado}%% Rótulo
     \includegraphics[scale=0.1, angle=-90]{figuras/Sistema_montado.jpg}%% Dimensões e localização
     \fonte{Autoria própria (2025)}%% Fonte
     \addcontentsline{loge}{figure}{\protect\numberline{\thefigure}Sistema montado e implementado}
 \end{figure}

A caixa do circuito deve ser instalada o mais alto possível, já a caixa do LCD e o botão devem ser fixados em um ponto relativamente alto, de forma a minimizar o contato com a água, mas não tão alto a ponto de serem inacessíveis pelo usuário.

\section{Desenvolvimento do \textit{software}}

Com o \textit{hardware} consolidado, o desenvolvimento do \textit{software}, além da funcionalidade, deve priorizar a experiência do usuário, gerando uma interface amigável e intuitiva. A implementação técnica foi executada na IDE Visual Studio Code, através do plugin PlatformIO, e ocorreu em duas frentes, \textit{firmware} e a aplicação \textit{web}.

\subsection{\textit{Firmware}}

O \textit{firmware} foi desenvolvido na linguagem C++ e implementa toda a lógica de aquisição, processamento e armazenamento dos dados, conforme ilustrado no fluxograma da Figura~\ref{fig:firmware_flowchart}. Na fase de inicialização, o sistema realiza a configuração dos periféricos (sensor de corrente SCT-013, sensor de vazão YF-S201 e a tela LCD), estabelece conexão à rede WiFi e carrega, a partir da memória flash interna do ESP32, os valores das tarifas de água e energia e o histórico de banhos. 
     
Durante a execução do loop principal, o firmware efetua leituras periódicas dos sensores e aplica critérios de detecção de evento: um banho é considerado iniciado quando a vazão de água excede 3 L/min. Nesse momento, o sistema transita para o estado ativo, no qual realiza amostragem contínua das grandezas, calcula o consumo acumulado de água e energia, atualiza os valores de custo em tempo real e os exibe no LCD. 

Quando o fluxo de água cai abaixo do limiar estabelecido, o sistema aguarda um período de até 5 minutos antes de finalizar oficialmente a sessão, caso novos sinais sejam detectados nesse intervalo, a sessão é mantida como contínua (funcionalidade de mesclagem de sessões). Após a confirmação do término, os dados da sessão são registrados na memória flash interna do ESP32. Para fins de visualização imediata, mantém-se uma memória circular com as cinquenta sessões mais recentes, armazenada na RAM do ESP32, utilizada no cálculo de estatísticas descritivas (média, mínimo, máximo), as quais são utilizadas pela interface LCD. 
     
Adicionalmente, foi implementado um mecanismo de reinício programado para ocorrer diariamente às 3h, visando mitigar possíveis instabilidades de longo prazo no microcontrolador, como possíveis vazamentos de memória. A medida de caráter preventivo, sem interferência nos dados coletados.

\begin{figure}[htbp]%% Ambiente figure
     %\captionsetup{width=0.55\textwidth}%% Largura da legenda
     \caption{Fluxograma do Firmware}%% Legenda
     \label{fig:firmware_flowchart}%% Rótulo
     \includegraphics[scale=0.18]{figuras/firmware_flowchart.png}%% Dimensões e localização
     \fonte{Autoria própria (2025)}%% Fonte
     \addcontentsline{loge}{figure}{\protect\numberline{\thefigure}Fluxograma do Firmware}
 \end{figure}
\FloatBarrier

O sistema continuamente monitora os sensores de vazão e corrente, e, ao detectar um banho, quantifica os sinais recebidos dos sensores em litros de água e em quilowatts-hora de energia, calculando a conversão monetária para apresentação em tempo real através da interface LCD e pela aplicação \textit{web}. Informações importantes como configurações e os dados históricos dos banhos são armazenadas na memória flash interna do microcontrolador, garantindo persistência mesmo em caso de reinicializações ou interrupções de energia.

Um sistema de reiniciamento automático foi implementado para atuar de forma preventiva em caso de vazamento de memória, assegurando a estabilidade do sistema em operação contínua.


\subsection{Aplicação \textit{web}}
 A opção por uma interface acessível via navegador foi motivada pela sua versatilidade. Ao contrário de um aplicativo instalado, a aplicação \textit{web} pode ser acessada em qualquer dispositivo com suporte a um navegador moderno, sem a necessidade de instalar \textit{softwares} adicionais. Isso torna o acesso mais fácil para os usuários e simplifica o processo de desenvolvimento.
 
Essa interface foi adotada como principal meio de interação com o sistema, oferecendo ao usuário tanto o acompanhamento em tempo real do consumo quanto a visualização histórica dos dados registrados. A Figura~\ref{fig:Web_Mockup} apresenta o protótipo inicial que orientou o desenvolvimento da interface

%A opção por uma interface acessível via navegador foi motivada pela sua versatilidade: ao contrário de aplicativos nativos, não exige instalação de software adicional e é compatível com qualquer dispositivo equipado com um navegador moderno — o que amplia o acesso e reduz barreiras técnicas para o usuário. Além disso, essa abordagem simplifica a manutenção e a atualização do sistema, já que as modificações são implementadas centralizadamente no servidor. 
     

 \begin{figure}[htbp]%% Ambiente figure
     %\captionsetup{width=0.55\textwidth}%% Largura da legenda
     \caption{Protótipo de interface}%% Legenda
     \label{fig:Web_Mockup}%% Rótulo
     \includegraphics[scale=0.15]{figuras/Web_Mockup.jpg}%% Dimensões e localização
     \fonte{Autoria própria (2025)}%% Fonte
     \addcontentsline{loge}{figure}{\protect\numberline{\thefigure}Protótipo de interface}
 \end{figure}

A interface \textit{web} foi implementada com tecnologias \textit{web} padrão, HTML, CSS e JavaScript, de forma semelhante a um site tradicional. O uso de JavaScript permite o processamento dos dados recebidos diretamente no navegador, aliviando a carga computacional do microcontrolador, que já é responsável pela análise contínua dos sensores.

 No painel, são mostradas as informações em tempo real sobre o consumo de água e energia elétrica do banho, assim como os custos do banho anterior.
 No final da página tem-se dois botões, sendo que o da esquerda força a atualização da página, e o da direita abre a aba de configuração.

\begin{figure}[htbp]%% Ambiente figure
     %\captionsetup{width=0.55\textwidth}%% Largura da legenda
     \caption{Painel}%% Legenda
     \label{fig:WebUi_1}%% Rótulo
     \includegraphics[scale=0.5]{figuras/WebUi_1.png}%% Dimensões e localização
     \fonte{Autoria própria (2025)}%% Fonte
     \addcontentsline{loge}{figure}{\protect\numberline{\thefigure}Painel}
 \end{figure}
Na aba de configuração, o usuário pode definir variáveis como o custo de energia, custo da água e o reset automático do sistema, mostrado na figura~\ref{fig:WebUi_cfg}.

\begin{figure}[htbp]%% Ambiente figure
     %\captionsetup{width=0.55\textwidth}%% Largura da legenda
     \caption{Página configuração}%% Legenda
     \label{fig:WebUi_cfg}%% Rótulo
     \includegraphics[scale=0.6]{figuras/WebUi_cfg.png}%% Dimensões e localização
     \fonte{Autoria própria (2025)}%% Fonte
     \addcontentsline{loge}{figure}{\protect\numberline{\thefigure}Página configuração}
 \end{figure}

 No histórico, ilustrado pelas Figuras~\ref{fig:WebUi_2} e Figura~\ref{fig:WebUi_3}, temos três gráficos de barras, ilustrando os custos dos banhos. 
 Em ordem, de cima para baixo:
 \begin{itemize}
     \item No topo tem-se o gráfico com os custos dos banhos do dia.
     \item No meio tem-se o gráfico com os custos de todos os banhos das últimas duas semanas.
     \item O terceiro e último gráfico mostra o custo médio de um banho nos últimos meses.
 \end{itemize}

 \begin{figure}[htbp]%% Ambiente figure
     %\captionsetup{width=0.55\textwidth}%% Largura da legenda
     \caption{Histórico parte superior}%% Legenda
     \label{fig:WebUi_2}%% Rótulo
     \includegraphics[scale=0.5]{figuras/WebUi_2.png}%% Dimensões e localização
     \fonte{Autoria própria (2025)}%% Fonte
     \addcontentsline{loge}{figure}{\protect\numberline{\thefigure}Histórico parte superior}
 \end{figure}

No final da página, encontra-se uma aba dedicada à visualização tabular dos registros de banho armazenados na memória flash interna do dispositivo. Diferentemente das representações gráficas anteriores, essa seção prioriza a precisão e a completude dos dados, apresentando, em formato de tabela, informações detalhadas de cada sessão registrada (data, duração, consumo de água e energia, custo total estimado). Adicionalmente, a interface permite a exportação integral desses registros em arquivo CSV, facilitando sua análise externa. 

 \begin{figure}[htbp]%% Ambiente figure
     %\captionsetup{width=0.55\textwidth}%% Largura da legenda
     \caption{Histórico parte inferior}%% Legenda
     \label{fig:WebUi_3}%% Rótulo
     \includegraphics[scale=0.5]{figuras/WebUi_3.png}%% Dimensões e localização
     \fonte{Autoria própria (2025)}%% Fonte
     \addcontentsline{loge}{figure}{\protect\numberline{\thefigure}Histórico parte inferior}
 \end{figure}

Dessa forma, além das funcionalidades oferecidas pela interface \textit{web}, o arquivo CSV disponibilizado permite ao usuário empregar ferramentas externas de análise para examinar os dados conforme suas próprias necessidades.

%falar do LCD
\subsection{Interface LCD}

Embora a aplicação \textit{web} proporcione flexibilidade no acesso remoto e no pós-processamento dos dados coletados, a obtenção de informações em tempo real exigiria que o usuário consultasse um dispositivo móvel (por exemplo, um smartphone) durante o banho, prática que se mostra pouco conveniente e, na maioria dos casos, inviável, dada a baixa resistência à umidade desses equipamentos. Diante disso, optou-se pela implementação de uma tela LCD com interface visual simplificada, capaz de fornecer indicação imediata do consumo de água e energia diretamente no ambiente do banheiro, garantindo assim acompanhamento direto e acessível das métricas relevantes ao usuário, sem exigir interação com dispositivos externos.

A interface na tela LCD é composta por dois modos: o primeiro, automaticamente exibido durante o banho, fornece o consumo do chuveiro elétrico em tempo real (Figura~\ref{fig:LCD_telaBanho}), e o segundo mostra um relatório pós-banho, comparando os custos do banho atual com a média dos custos dos banhos anteriores (Figura~\ref{fig:LCD_telaPósBanho}), a tela se apaga automaticamente após um curto período de inatividade do sistema. Em ambos os modos, o endereço de acesso à interface \textit{web} é exibido na região inferior da tela.

%O endereço de acesso à interface \textit{web} por meio de um navegador é visível em ambas as partes na seção inferior.

 \begin{figure}[htbp]%% Ambiente figure
     %\captionsetup{width=0.55\textwidth}%% Largura da legenda
     \caption{\textit{Feedback} via interface na tela LCD}%% Legenda
     \label{fig:LCD_telaBanho}%% Rótulo
     \includegraphics[scale=0.15]{figuras/LCD_telaBanho.jpg}%% Dimensões e localização
     \fonte{Autoria própria (2025)}%% Fonte
     \addcontentsline{loge}{figure}{\protect\numberline{\thefigure}\textit{Feedback} via interface na tela LCD}
 \end{figure}



 \begin{figure}[htbp]%% Ambiente figure
     %\captionsetup{width=0.55\textwidth}%% Largura da legenda
     \caption{Relatório pós banho}%% Legenda
     \label{fig:LCD_telaPósBanho}%% Rótulo
     \includegraphics[scale=0.08]{figuras/LCD_telaPósBanho.jpg}%% Dimensões e localização
     \fonte{Autoria própria (2025)}%% Fonte
     \addcontentsline{loge}{figure}{\protect\numberline{\thefigure}Relatório pós banho}
 \end{figure}

\chapter{Resultados e discussões}
%Resultados PRATICOS
Os primeiros resultados obtidos por meio da montagem e operação do projeto são apresentados nessa seção. Aqui se tem por objetivo a visualização e validação dos dados coletados, assim como os resultados dos mesmos.

\section{Validação Experimental}
%desenrolar sobre como as leituras foram validadas
Apresenta-se a seguir o passo a passo do uso de ferramentas para validação das leituras obtidas pelo sistema, garantindo seu correto funcionamento.

\subsection{Ferramentas utilizadas}

Para medir a vazão real do chuveiro, foram usados um balde, uma balança digital (ilustrados na Figura~\ref{fig:balança_e_balde}) e um cronômetro. O método consistiu em coletar água do chuveiro no balde durante um intervalo de tempo cronometrado, e a massa de água coletada pelo balde foi registrada pela balança.

 \begin{figure}[htbp]%% Ambiente figure
     %\captionsetup{width=0.55\textwidth}%% Largura da legenda
     \caption{Balança digital e balde}%% Legenda
     \label{fig:balança_e_balde}%% Rótulo
     \includegraphics[scale=0.1]{figuras/balança_e_balde.jpg}%% Dimensões e localização
     \fonte{Autoria própria (2025)}%% Fonte
     \addcontentsline{loge}{figure}{\protect\numberline{\thefigure}Balança digital e balde}
 \end{figure}

Para coleta dos dados da corrente no circuito, foi empregado o alicate amperímetro ilustrado na Figura~\ref{fig:amperimetro}.

 \begin{figure}[htbp]%% Ambiente figure
     %\captionsetup{width=0.55\textwidth}%% Largura da legenda
     \caption{Alicate amperímetro}%% Legenda
     \label{fig:amperimetro}%% Rótulo
     \includegraphics[scale=0.1]{figuras/amperimetro.jpg}%% Dimensões e localização
     \fonte{Autoria própria (2025)}%% Fonte
     \addcontentsline{loge}{figure}{\protect\numberline{\thefigure}Alicate amperímetro}
 \end{figure}

\subsection{Validação dos valores medidos}

A partir dos dados coletados de diversos testes realizados, duas tabelas foram organizadas: a Tabela~\ref{tab:tabCorrente}, com as medições de corrente elétrica, e a Tabela~\ref{tab:tabVazao}, com os valores de vazão de água. Em ambas, a primeira coluna contém os valores registrados pelo sistema desenvolvido neste trabalho; a segunda coluna tem os valores de referência medidos manualmente com as ferramentas de medição; a terceira coluna apresenta o erro percentual entre as duas leituras, calculado pela Equação~\ref{eq:erro_percentual}, utilizado para avaliar a precisão do sistema. Adicionalmente, a Tabela~\ref{tab:calculoVazao} é utilizada para organizar os dados calculados.

\begin{equation}
\label{eq:erro_percentual}
E_{\%} = \left| \frac{V_m - V_r}{V_r} \right| \times 100
\end{equation}

\noindent onde:
\begin{itemize}
    \item $E_{\%}$ é o erro percentual (\%);
    \item $V_m$ é o valor medido;
    \item $V_r$ é o valor de referência.
\end{itemize}

A Tabela~\ref{tab:tabCorrente} apresenta a comparação entre os valores de corrente registrados pelo sistema proposto na primeira coluna e aqueles obtidos experimentalmente por meio de um alicate amperímetro de referência na segunda coluna, cujo modelo é ilustrado na Figura~\ref{fig:amperimetro}. As medições foram realizadas em um chuveiro elétrico com duas posições de potência: verão e inverno, alternadas durante os ensaios.

\begin{table}[!htb]
 \captionsetup{width=0.83\textwidth}
 \centering
 \caption{\label{tab:tabCorrente}Leitura de corrente}
 \begin{tabular}{ccc}
 \hline
 \multicolumn{1}{c}{Leituras registradas (A)} & Valores de referência (A) & Erro percentual (\%) \\
 \hline
 25.56 & 25.48 & 0.31 \\
 25.47 & 25.39 & 0.31 \\
 25.61 & 25.52 & 0.35 \\
 25.41 & 25.31 & 0.39 \\
 25.56 & 25.41 & 0.59 \\
 41.98 & 40.86 & 2.74 \\
 42.05 & 40.89 & 2.83 \\
 42.22 & 41.02 & 2.92 \\
 41.76 & 40.51 & 3.08 \\
 42.11 & 40.79 & 3.23 \\
 42.02 & 40.57 & 3.57 \\
 \hline
 \end{tabular}
 \fonte{Autoria própria (2025)}
 \end{table}
\FloatBarrier
Como se pode ver, os valores de corrente registrados pelo sistema são bem próximos dos valores medidos, com o maior erro sendo menor que 4\%. 

Antes de serem tabelados, os valores de vazão foram calculados, conforme ilustrado na Tabela~\ref{tab:calculoVazao}, por meio da Equação~\ref{eq:vazao}. Nota-se que, para calcular a vazão com a Equação~\ref{eq:vazao}, foi considerada a densidade da água aproximadamente igual a 1 g/mL, e a massa de água coletada foi diretamente convertida em volume, de modo que 1 grama corresponde a 1 mililitro.

\begin{equation}
\label{eq:vazao}
Q =  \frac{V_l}{t}  \times 60
\end{equation}

\noindent onde:
\begin{itemize}
    \item $Q$ é a vazão (L/min);
    \item $V_l$ é o volume de água medido (L);
    \item $t$ é o intervalo de tempo durante o qual o volume de água foi coletado (em segundos).
\end{itemize}

 \begin{table}[!htb]
 % Luiz - O texto do caption da tabela/quadro deve ser do tamanho da tabela, então utilize a linha a seguir para conseguir esse efeito
 \captionsetup{width=0.83\textwidth}
 \centering
 \caption{\label{tab:calculoVazao}Vazão calculada}
 \begin{tabular}{ccc}
 \hline
\multicolumn{1}{c}{Volume de água (L)}     & tempo (segundos) & Vazão calculada (L/min) \\ 
\hline
 1.96 & 45.22 & 2.6 \\
 1.76 & 31.34 & 3.38 \\
 1.35 & 25.66 & 3.16 \\
 1.95 & 40.63 & 2.88 \\
 2.16 & 43.38 & 2.98 \\
 1.76 & 46.06 & 2.3 \\
 1.83 & 50.1 & 2.19 \\
 0.89 & 32.29 & 1.66 \\
 1.59 & 53.37 & 1.79 \\
 1.31 & 72.44 & 1.08 \\
 \hline
 \end{tabular}
 \fonte{Autoria própria (2025)}
 \end{table}

Os valores de vazão calculados na Tabela~\ref{tab:calculoVazao} foram utilizados como referência na Tabela~\ref{tab:tabVazao}, permitindo uma comparação direta com as medições registradas pelo sistema.

\begin{table}[!htb]
 \captionsetup{width=0.83\textwidth}
 \centering
 \caption{\label{tab:tabVazao}Leitura de vazão}
 \begin{tabular}{ccc}
 \hline
\multicolumn{1}{c}{Leituras registradas (L/min)}     & Valores de referência (L/min) & Erro percentual (\%) \\
 \hline
 2.60 & 2.60 & 0.13 \\
 3.42 & 3.38 & 1.27 \\
 3.20 & 3.16 & 1.30 \\
 2.95 & 2.88 & 2.44 \\
 3.07 & 2.98 & 2.95 \\
 2.16 & 2.30 & 5.95 \\
 1.97 & 2.19 & 9.87 \\
 1.19 & 1.66 & 28.12 \\
 1.21 & 1.79 & 32.35 \\
 0.54 & 1.08 & 50.12 \\
 \hline
 \end{tabular}
 \fonte{Autoria própria (2025)}
\end{table}
\FloatBarrier

Observa-se, pelos dados apresentados na terceira coluna da Tabela~\ref{tab:tabVazao}, que o sensor de vazão apresenta redução significativa de precisão em regimes de baixo fluxo. Especificamente, para vazões inferiores a 3 L/min, os desvios medidos tornam-se muito elevados, evidenciando uma limitação do sensor empregado. Consequentemente, recomenda-se que o sistema seja empregado somente em condições cuja vazão mínima exceda esse limiar, sob pena de comprometimento da confiabilidade das medições.

O sensor de corrente SCT-013 demonstrou consistência ao longo das medições, com desvios máximos vistos na Tabela~\ref{tab:tabCorrente} de 3,57\% em relação ao valor esperado. 
%%O erro VAI impactar a aplicação do sistema


\section{Análise dos Dados Coletados pelo Sistema}

O sistema foi operado e monitorado ao longo de um período de vários dias, conforme ilustrado no gráfico da Figura~\ref{fig:WebPrimeirasDuasSemanas}, que apresenta o custo total diário dos banhos registrados. Esses dados são disponibilizados ao usuário em tempo real, tanto por meio da aplicação \textit{web} quanto da interface LCD local, com o intuito exclusivo de fornecer transparência sobre o consumo. Cabe ao próprio usuário interpretar as informações e decidir, de forma autônoma, sobre eventuais ajustes em seus hábitos.

 \begin{figure}[htbp]%% Ambiente figure
     %\captionsetup{width=0.55\textwidth}%% Largura da legenda
     \caption{Gráfico de custo diário ao longo de vários dias}%% Legenda
     \label{fig:WebPrimeirasDuasSemanas}%% Rótulo
     \includegraphics[scale=0.70]{figuras/WebPrimeirasDuasSemanas.png}%% Dimensões e localização
     \fonte{Autoria própria (2025)}%% Fonte
     \addcontentsline{loge}{figure}{\protect\numberline{\thefigure}Gráfico de custo diário ao longo de vários dias}
 \end{figure}

\section{Limitações técnicas}
%falar que a potência é estimada por causa da falta de leitura da  tensão
A potência elétrica apresentada corresponde a uma estimativa, uma vez que o cálculo pressupõe o valor de tensão eficaz da rede como constante, desconsiderando eventuais flutuações. Essa simplificação foi adotada intencionalmente como compromisso projetual, visando reduzir custos, complexidade e o número de componentes, sem comprometer a utilidade funcional do sistema para fins de conscientização do usuário. 
     
Quanto ao sensor de corrente (SCT-013), verificou-se sensibilidade à posição de instalação em torno do condutor: pequenas variações na leitura ocorreram conforme sua orientação angular e centralização no cabo. Essas oscilações aparentam ser de baixa magnitude, mas um estudo mais profundo deve ser conduzido de forma a isolar quantitativamente a contribuição do posicionamento em relação a outras fontes de incerteza. 
     
Por fim, o sensor de vazão (YF-S201) exibiu degradação acentuada de desempenho em regimes de fluxo abaixo de três litros por minuto, com aumento expressivo do erro relativo. Tal comportamento inviabiliza sua aplicação em chuveiros de baixa vazão, constituindo uma limitação relevante do escopo operacional do sistema proposto. 
     
