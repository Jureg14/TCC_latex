%%%% CAPÍTULO 5 - CONCLUSÕES E PERSPECTIVAS
%%
\chapter{Conclusão Parcial}\label{cap:conclusoeseperspectivas}

%Parte final do texto, na qual se apresentam as conclusões do trabalho acadêmico, usualmente denominada Considerações Finais. Pode ser usada outra denominação similar que indique a conclusão do trabalho.

O trabalho teve como objetivo principal o desenvolvimento de um sistema embarcado de baixo custo para o monitoramento simultâneo do consumo de água e de energia em chuveiros elétricos. Seu propósito é fornecer ao usuário acesso transparente e em tempo real aos dados de consumo associados ao uso do chuveiro, possibilitando maior clareza sobre o impacto de seus hábitos domésticos. 


Para atingir este objetivo, foi desenvolvido um protótipo funcional baseado no microcontrolador ESP32 , utilizando o sensor de corrente SCT-013-000 e o sensor de vazão YF-S201 para a coleta de dados. Além dos sensores, de forma a interagir com o usuário, foram desenvolvidas duas interfaces de usuário: uma local, composta por um display LCD conectado diretamente ao ESP32, e outra remota, na forma de uma aplicação \textit{web} acessível por dispositivos conectados à mesma rede. 

O sistema foi instalado e operado em ambiente residencial real por vários dias. Os dados coletados evidenciam sua capacidade de registrar variações diárias de consumo e de disponibilizar essas informações de forma contínua. Como ilustrado na Figura~\ref{fig:WebPrimeirasDuasSemanas}, observam-se oscilações no custo diário dos banhos, incluindo períodos de redução, o que demonstra a sensibilidade do sistema à variação no comportamento do usuário, sem, contudo, estabelecer causalidade entre o monitoramento e a apresentação dos dados e eventuais mudanças.

Das limitações técnicas presentes no trabalho, destacam-se: a tensão da rede é assumida como constante, o que faz com que a potência medida seja uma estimativa, e o sensor de vazão utilizado apresenta uma grande imprecisão em condições de baixo fluxo. 
     


\chapter{Cronograma}\label{cap:cronograma}

Este capítulo apresenta o cronograma das atividades a serem desenvolvidas para a conclusão do Trabalho de Conclusão de Curso 2, conforme descrito no Quadro~\ref{quad:cronograma}.

\begin{tabframed}[htb]
\caption{Cronograma de atividades para TCC 2}
\label{quad:cronograma}
\renewcommand{\arraystretch}{1.5}
\begin{tabular}{|p{7cm}|c|c|c|c|c|c|}
\hline
\multicolumn{1}{|c|}{\textbf{Atividade}} & \textbf{Jan} & \textbf{Fev} & \textbf{Mar} & \textbf{Abr} & \textbf{Mai} & \textbf{Jun} \\ \hline
Aprimoramento do sistema & X & & & & & \\ \hline
Calibração dos sensores & X & X & & & & \\ \hline
Otimização do firmware & & X & X & & & \\ \hline
Melhorias na interface \textbf{web} & & X & X & & & \\ \hline
Testes de longa duração & & & X & X & & \\ \hline
Coleta de dados em campo & & & X & X & X & \\ \hline
Análise estatística dos dados & & & & X & X & \\ \hline
Redação dos resultados & & & & X & X & \\ \hline
Redação da conclusão & & & & & X & X \\ \hline
Revisão bibliográfica final & & & & & X & X \\ \hline
Formatação ABNT & & & & & & X \\ \hline
Revisão ortográfica & & & & & & X \\ \hline
Preparação da apresentação & & & & & & X \\ \hline
Defesa do TCC 2 & & & & & & X \\ \hline
\end{tabular}
\fonte{Autoria própria (2025)}
\addcontentsline{loge}{tabframed}{\protect\numberline{\thetabframed}Cronograma de atividades para TCC 2}
\end{tabframed}