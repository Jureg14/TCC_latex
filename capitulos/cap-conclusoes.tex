%%%% CAPÍTULO 5 - CONCLUSÕES E PERSPECTIVAS
%%
\chapter{Conclusão}\label{cap:conclusoeseperspectivas}

%Parte final do texto, na qual se apresentam as conclusões do trabalho acadêmico, usualmente denominada Considerações Finais. Pode ser usada outra denominação similar que indique a conclusão do trabalho.

O trabalho apresentado teve como principal objetivo o desenvolvimento de um sistema embarcado de baixo custo para o monitoramento do consumo de água e energia em chuveiros elétricos. Seu propósito é de fornecer ao usuário um método acessível de monitorar o impacto de seus hábitos em tempo real, incentivando a adoção de práticas mais sustentáveis.

De forma a atingir este objetivo, foi desenvolvido um protótipo funcional baseado no microcontrolador ESP32 , utilizando o sensor de corrente SCT-013 e o sensor de vazão YF-S201 para coleta dos dados. Além dos sensores, de forma a interagir com o usuário, foram desenvolvidas duas interfaces, uma simplificada, utilizando de uma LCD acoplada ao ESP32, e outra mais completa, feita na forma de Web App.


O sistema foi instalado e monitorado em um ambiente real por um período de duas semanas, os dados coletados corroboraram a hipótese central do trabalho: o feedback em tempo real contribui efetivamente para a conscientização do usuário. Como ilustrado na Figura~\ref{fig:WebPrimeirasDuasSemanas}, se observa uma tendência de queda nos gastos diários.
%O sistema foi instalado e monitorado em um ambiente real por um período de duas semanas, os dados coletados corroboraram a hipótese central do trabalho: o feedback em tempo real contribui efetivamente para a conscientização do usuário. Como ilustrado na Figura~\ref{fig:WebPrimeirasDuasSemanas}, se observa uma tendência de queda nos gastos diários, caindo de uma média de R\$7 nos primeiros dias para cerca de R\$5 ao final dos testes.

Das limitações técnicas presentes no trabalho, destacam-se: a tensão da rede é assumida como constante, o que faz com que a potência medida seja uma estimativa, e o sensor de vazão utilizado apresenta uma grande imprecisão em condições de baixo fluxo. Para trabalhos futuros, recomenda-se a integração de um sensor de tensão, permitindo o cálculo da potência real em vez da estimada.