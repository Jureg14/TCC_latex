%%%% CAPÍTULO 5 - CONCLUSÕES E PERSPECTIVAS
%%
\chapter{Conclusão}\label{cap:conclusoeseperspectivas}

%Parte final do texto, na qual se apresentam as conclusões do trabalho acadêmico, usualmente denominada Considerações Finais. Pode ser usada outra denominação similar que indique a conclusão do trabalho.

O trabalho apresentado teve como principal objetivo o desenvolvimento de um sistema embarcado de baixo custo para o monitoramento do consumo de água e energia em chuveiros elétricos. Seu propósito é de fornecer ao usuário um método acessível de monitorar o impacto de seus hábitos em tempo real, incentivando a adoção de práticas mais sustentáveis.
%O presente trabalho teve como objetivo principal o desenvolvimento de um sistema embarcado de baixo custo para o monitoramento do consumo de água e energia em chuveiros elétricos. A finalidade do sistema é fornecer ao usuário um feedback em tempo real e um histórico de gastos , promovendo a conscientização sobre o impacto de seus hábitos e incentivando a adoção de práticas mais sustentáveis.%IA

De forma a atingir este objetivo, foi desenvolvido um protótipo funcional baseado no microcontrolador ESP32 , utilizando do sensor de corrente SCT-013 e do sensor de vazão YF-S201 para coleta dos dados. Além dos sensores, de forma a interagir com o usuário, foram desenvolvidas duas interfaces, uma simplificada, utilizando de uma LCD acoplada ao ESP32, e outra mais completa, feita na forma de Web App.
%Para atingir este objetivo, foi desenvolvido um protótipo funcional baseado no microcontrolador ESP32 , equipado com o sensor de corrente SCT-013 e o sensor de vazão YF-S201. A solução se completa com uma interface Web App para visualização e configuração dos dados. A validação experimental demonstrou a viabilidade da solução: o sensor de corrente obteve leituras com um erro absoluto máximo inferior a 3\%. O sensor de vazão, embora tenha apresentado erros significativos em fluxos baixos , mostrou-se satisfatório na faixa de operação típica do chuveiro (acima de 2,5 L/min), onde o erro foi considerado aceitável.%IA

O sistema foi instalado e monitorado em um ambiente real por um período de duas semanas, os dados coletados corroboraram a hipótese central do trabalho: o feedback em tempo real contribui efetivamente para a conscientização do usuário. Como ilustrado na Figura~\ref{fig:WebPrimeirasDuasSemanas}, se observa uma tendência de queda nos gastos diarios, caindo de uma média de R\$7 nos primeiros dias para cerca de R\$5 ao final dos testes.
%O sistema foi instalado e monitorado em um ambiente real por um período de duas semanas. Os dados coletados comprovaram a principal hipótese do trabalho: o feedback em tempo real é eficaz em promover a conscientização. Conforme ilustrado na Figura 16 , foi observada uma clara tendência de queda nos gastos diários, que se reduziram de uma média de R\$7 nos primeiros dias para cerca de R\$5 ao final do período de testes. Este resultado demonstra o sucesso do protótipo em seu objetivo de incentivar a redução de desperdícios.%IA

Das limitações tecnicas presentes no trabalho, destacam-se: a tensão da rede é assumida como constante, o que faz com que a potência medida seja uma estimativa, e o sensor de vazão utilizado apresenta uma grande imprecisão em condições de baixo fluxo. Para trabalhos futuros, recomenda-se a integração de um sensor de tensão, permitindo o cálculo da potência real em vez da estimada.
%Como limitações técnicas, destaca-se que o cálculo da potência é uma estimativa, pois o sistema assume a tensão da rede como um valor constante. Além disso, o sensor de vazão adotado demonstrou imprecisão em condições de baixo fluxo. Para trabalhos futuros, sugere-se a incorporação de um sensor de tensão para permitir o cálculo da potência real, em vez da estimada. Recomenda-se também a realização de um estudo de caso com um grupo maior de usuários para analisar os efeitos do sistema na mudança de hábitos em maior escala. %IA