\chapter{FUNDAMENTAÇÃO TEÓRICA}

Nesta seção são apresentados os pilares teóricos sobre os quais o trabalho é estruturado.

\section{Sistemas Embarcados e o Microcontrolador ESP32} 

Sistemas embarcados são combinações integradas de \textit{hardware} e \textit{software} projetadas para executar uma ou poucas funções específicas dentro de um sistema maior, diferentemente de computadores de propósito geral, que operam em ambientes versáteis e multifuncionais. São amplamente empregados em aplicações cotidianas, como eletrodomésticos, automóveis e dispositivos médicos, onde requisitos como confiabilidade, baixo custo e consumo energético reduzido são prioritários \cite[p.~1]{BARR:2006}. 
%  \textit{web} e LCD

O microcontrolador é o componente central da maioria dos sistemas embarcados, consistindo em um computador completo integrado em um único chip, que incorpora unidade central de processamento , memória  e periféricos de entrada e saída. Essa integração permite o desenvolvimento de soluções compactas, eficientes e economicamente viáveis, especialmente em aplicações que exigem aquisição de dados de sensores, controle de atuadores ou automação de processos \cite[p.~35]{VALDESPEREZ:2013}.

Dentre os microcontroladores disponíveis no mercado, o ESP32 destaca-se como um sistema em um chip (SoC, do inglês \textit{System on a Chip}), desenvolvido pela Espressif Systems. Além de recursos clássicos de um microcontrolador, o ESP32 integra conectividade sem fio com suporte nativo a redes WiFi e Bluetooth, o que o torna especialmente adequado para aplicações de IoT, onde comunicação remota e monitoramento em tempo real são requisitos comuns \cite{ESPRESSIF:2016,ESPRESSIF:ESP32DS}. 

\begin{comment}
\section{Sistemas Embarcados}

Um sistema embarcado é uma combinação integrada de \textit{hardware} e \textit{software}, sendo projetada para uma função específica. Em contraste com os computadores de uso pessoal, que atuam executando uma ampla gama de tarefas, os sistemas embarcados são criados de forma a executarem uma tarefa em específico, sendo frequentemente utilizados como parte de um sistema maior, como eletrodomésticos e automóveis \cite[p. 1]{BARR:2006}.

\subsection{Microcontroladores}

O microcontrolador consiste em um computador completo em um chip, contendo um processador, memória e interfaces de entrada/saída em um sistema integrado. Essa arquitetura integrada permite criar soluções compactas e de baixo custo para aplicações específicas, sendo o cérebro de grande parte dos sistemas embarcados onde o controle de dispositivos é necessário, atuando no processamento de dados de sensores ou automatizando tarefas \cite{VALDESPEREZ:2013}.

\subsubsection{ESP32}

O microcontrolador ESP32 é um sistema em um chip (SoC, do inglês \textit{System on a Chip}) de baixo custo desenvolvido pela Espressif Systems, projetado para uma vasta gama de aplicações, desde redes de sensores de baixa potência até tarefas complexas, sendo um componente central na prototipagem e desenvolvimento de projetos de Internet das Coisas (IoT) \cite{ESPRESSIF:2016}. Trata-se de um chip híbrido que integra conectividade WiFi (802.11 b/g/n) e Bluetooth em um único módulo \cite{ESPRESSIF:ESP32DS}.
\end{comment}

\section{Internet das Coisas} %ok

A Internet das Coisas (IoT) pode ser compreendida como uma rede na qual objetos físicos podem formar conexões entre si. Esses objetos podem ser equipados com sensores, atuadores e módulos de comunicação sem fio, interagindo entre si e com servidores centrais, formando um sistema inteligente e versátil, o qual pode ser aplicado em diversos setores, como agricultura, indústria, logística e automação residencial \cite{ATZORI20102787}.

\section{Medição de Energia Elétrica} 

A medição de energia elétrica em sistemas de corrente alternada envolve o entendimento de três grandezas fundamentais que compõem a potência elétrica: a potência ativa, a potência reativa e a potência aparente. A potência ativa, medida em quilowatts, representa a parcela da energia efetivamente convertida em trabalho útil, como calor, luz ou movimento. A potência reativa, expressa em quilovolt-ampères reativos, está associada à criação e manutenção de campos magnéticos em cargas indutivas, como motores e transformadores, mas não realiza trabalho útil. A potência aparente, em quilovolt-ampères, é a resultante vetorial dessas duas componentes e indica a potência total fornecida pela rede \cite{WILDI:2013}. Embora presente em todos os consumidores, a potência reativa só é tarifada para consumidores de alta tensão \cite{ANEEL:REN1000_2021}. 

Transformadores são dispositivos eletromagnéticos estáticos que permitem a transferência de energia entre circuitos elétricos por meio da indução eletromagnética, sem conexão elétrica direta entre eles. São constituídos por dois ou mais enrolamentos condutores, geralmente sobre um núcleo ferromagnético comum, sendo o enrolamento primário conectado à fonte de alimentação e o secundário à carga. Sua principal função é elevar ou reduzir os níveis de tensão em sistemas de corrente alternada, mantendo constante a frequência do sinal \cite[p.~189]{WILDI:2013}. Conforme ilustrado na Figura~\ref{fig:transIdeal}, o enrolamento primário é caracterizado pela corrente $i_p(t)$ , tensão $V_p(t)$  e número de espiras $N_p$ , enquanto o enrolamento secundário é definido por $i_s(t)$ , $V_s(t)$  e $N_s$ , respectivamente.

\begin{figure}[!htb]%% Ambiente figure
     %\captionsetup{width=0.55\textwidth}%% Largura da legenda
     \caption{Desenho esquemático de um transformador ideal}%% Legenda
     \label{fig:transIdeal}%% Rótulo
     \includegraphics[scale=0.45]{transformador_basico_Chapman}%% Dimensões e localização
     \fonte{Adaptado de \cite{CHAPMAN:2013}}%% Fonte
     \addcontentsline{loge}{figure}{\protect\numberline{\thefigure}Desenho esquemático de um transformador ideal}
     %\addcontentsline{loge}{geralgraph}{\protect\numberline{\thegeralgraph}Fluxograma do Firmware}
 \end{figure}
\FloatBarrier

Entre os tipos especializados, os transformadores de corrente (TCs) são projetados para medir com precisão a corrente alternada que circula em um circuito, sendo conectados em série com a carga. Devido à sua função de medição e proteção, operam com baixa potência, geralmente inferior a 200 VA \cite[p.~189]{WILDI:2013}. Um modelo comum em aplicações práticas é o transformador de corrente com núcleo partido, ilustrado na Figura~\ref{fig:transUsado}, cuja estrutura permite instalação rápida em condutores já energizados, sem necessidade de interrupção do fornecimento de energia, característica que favorece sua adoção em sistemas de monitoramento não invasivo \cite{SIEMENS:2015}. 

\begin{figure}[!htb]%% Ambiente figure
     %\captionsetup{width=0.55\textwidth}%% Largura da legenda
     \caption{Transformador de corrente de nucleo bipartido}%% Legenda
     \label{fig:transUsado}%% Rótulo
     \includegraphics[scale=0.1]{transformador_nucleo_bipartido}%% Dimensões e localização
     \fonte{Autoria Própria}%% Fonte
     \addcontentsline{loge}{figure}{\protect\numberline{\thefigure}Transformador de corrente de nucleo bipartido}
          %\addcontentsline{loge}{geralgraph}{\protect\numberline{\thefigure}transformador de corrente de nucleo bipartido}
 \end{figure}

\begin{comment}

\section{Medição de Energia Elétrica} 

A medição de energia elétrica é relevante para o controle do fluxo de energia nos sistemas de distribuição. Sendo assim, na corrente alternada, a energia é constituída por três elementos fundamentais que compõem a potência elétrica: potência ativa, potência reativa e potência aparente.

A potência ativa é a parte da energia que pode ser convertida em trabalho útil (calor ou luz) e que é medida em quilowatts (kW). A potência reativa se expressa em quilovolt-ampères reativos (kVAR) e é aquela que requer energia para criar e manter, em equipamentos indutivos (motor, transformador, dentre outros), os campos magnéticos de que necessitam para operar. A potência aparente (kVA) é a combinação vetorial entre a potência ativa e reativa e representa a potência fornecida pela rede \cite{WILDI:2013}.

O faturamento de energia elétrica do setor residencial é feito apenas em função da potência ativa consumida. Já a potência reativa, apesar de estar presente em todos os consumidores, em diversos graus de magnitude, é faturada somente para os consumidores de alta tensão, sobre os quais serão cobradas multas para o consumo de potência reativa acima do limite \cite{ANEEL:2000}.

\subsection{Transformadores} %ok

O transformador é um dispositivo eletromagnético estático que opera com base no princípio da indução eletromagnética, permitindo a transferência de energia elétrica entre dois ou mais circuitos sem conexão elétrica direta. Sua principal função é elevar ou reduzir os níveis de tensão em sistemas de corrente alternada (CA), mantendo inalterada a frequência do sinal \cite[p. 189]{WILDI:2013}. Conforme ilustrado na Figura~\ref{fig:transIdeal}, o enrolamento primário é caracterizado pela corrente $i_p(t)$ , tensão $V_p(t)$  e número de espiras $N_p$ , enquanto o enrolamento secundário é definido por $i_s(t)$ , $V_s(t)$  e $N_s$ , respectivamente.

\begin{figure}[!htb]%% Ambiente figure
     %\captionsetup{width=0.55\textwidth}%% Largura da legenda
     \caption{Desenho esquemático de um transformador ideal}%% Legenda
     \label{fig:transIdeal}%% Rótulo
     \includegraphics[scale=0.45]{transformador_basico_Chapman}%% Dimensões e localização
     \fonte{Adaptado de \cite{CHAPMAN:2013}}%% Fonte
     \addcontentsline{loge}{figure}{\protect\numberline{\thefigure}Desenho esquemático de um transformador ideal}
     %\addcontentsline{loge}{geralgraph}{\protect\numberline{\thegeralgraph}Fluxograma do Firmware}
 \end{figure}
\FloatBarrier

Estruturalmente, o transformador é composto por duas ou mais bobinas condutoras enroladas sobre um núcleo comum de material ferromagnético. Em geral, essas bobinas não apresentam conexão elétrica direta entre si, sendo o acoplamento realizado exclusivamente através do fluxo magnético compartilhado no interior do núcleo.

O enrolamento conectado à fonte de energia elétrica é denominado enrolamento primário ou de entrada, enquanto o enrolamento responsável por fornecer energia à carga é denominado enrolamento secundário ou de saída. Quando presente, um terceiro enrolamento recebe a designação de enrolamento terciário, sendo empregado em aplicações específicas, conforme a finalidade do transformador \cite[p. 66]{CHAPMAN:2013}.

\subsubsection{Transformadores de corrente}

São transformadores especializados utilizados em série com um circuito elétrico para medir com precisão a corrente alternada que flui pelo circuito. Devido à natureza de seu uso, como medição e proteção de sistemas, a potência desses transformadores costuma ser pequena, normalmente abaixo de 200 VA \cite[p. 189]{WILDI:2013}.
A variação com núcleo partido, ilustrada na Figura~\ref{fig:transUsado}, é particularmente interessante, pois pode ser facilmente instalada em um sistema existente sem necessidade de desconectar a linha ou interromper o fornecimento de energia \cite{SIEMENS:2015}.

\begin{figure}[!htb]%% Ambiente figure
     %\captionsetup{width=0.55\textwidth}%% Largura da legenda
     \caption{Transformador de corrente de nucleo bipartido}%% Legenda
     \label{fig:transUsado}%% Rótulo
     \includegraphics[scale=0.1]{transformador_nucleo_bipartido}%% Dimensões e localização
     \fonte{Autoria Própria}%% Fonte
     \addcontentsline{loge}{figure}{\protect\numberline{\thefigure}Transformador de corrente de nucleo bipartido}
          %\addcontentsline{loge}{geralgraph}{\protect\numberline{\thefigure}transformador de corrente de nucleo bipartido}
 \end{figure}
\end{comment}


\section{Medição do Fluxo de Água} 

A medição do fluxo de água é um processo essencial em diversas áreas, desde o manejo de recursos hídricos até sistemas de distribuição urbana, e envolve diferentes abordagens conforme o contexto físico e as condições de escoamento. 

Em cursos d’água abertos, como rios e riachos, métodos clássicos são utilizados para estimar a vazão, geralmente com base na relação entre a velocidade média do fluxo e a área da seção transversal do canal. Exemplos incluem o uso do medidor de molinete, o método do flutuador, vertedores calibrados e sensores ultrassônicos de nível, que inferem a vazão indiretamente a partir de medições de altura ou velocidade superficial \cite{COSTA_Camargo_Tolentino_Akutsu_Periotto_Tanaka:2023}. 

Em sistemas fechados, nos quais o líquido escoa por tubulações, como redes de abastecimento, estações de tratamento ou instalações prediais, técnicas alternativas são empregadas. Entre elas, destacam-se os hidrômetros mecânicos, sensores eletromagnéticos, que se baseiam na lei da indução de Faraday, e medidores ultrassônicos, que utilizam o efeito Doppler ou o princípio de tempo de trânsito para determinar a velocidade do fluido \cite{flow:2016}. 

O sensor YF-S201, ilustrado na Figura~\ref{fig:YF-S201}, é um dispositivo de medição de vazão volumétrica para líquidos.  Em seu núcleo, uma turbina contendo um ímã permanente é posta em rotação pelo fluxo do líquido. A cada rotação completa, o ímã passa próximo a um sensor de efeito Hall integrado ao corpo do dispositivo, o que gera um pulso elétrico de saída. Dado que a frequência desta sequência de pulsos é linearmente proporcional à velocidade do fluxo, a vazão instantânea pode ser determinada. O volume total escoado é subsequentemente calculado pela integração dos pulsos ao longo do tempo, utilizando o fator de calibração característico do sensor \cite{YF-S201:datasheet}.


\begin{figure}[!htb]%% Ambiente figure
     %\captionsetup{width=0.55\textwidth}%% Largura da legenda
     \caption{Sensor YF-S201}%% Legenda
     \label{fig:YF-S201}%% Rótulo
     \includegraphics[scale=0.09]{figuras/YF-S201.jpg}%% Dimensões e localização
     \fonte{Autoria Própria}%% Fonte
     \addcontentsline{loge}{figure}{\protect\numberline{\thefigure}Sensor YF-S201}
 \end{figure}
