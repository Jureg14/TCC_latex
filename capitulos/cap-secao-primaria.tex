\chapter{FUNDAMENTAÇÃO TEÓRICA}

Nesta seção são apresentados os pilares teóricos sobre os quais o trabalho é estruturado.

\section{Sistemas Embarcados}

Um sistema embarcado é uma combinação integrada de \textit{hardware} e \textit{software}, sendo projetada para uma função específica. Em contraste com os computadores de uso pessoal, que atuam executando uma ampla gama de tarefas, os sistemas embarcados são criados de forma a executarem uma tarefa em específico, sendo frequentemente utilizados como parte de um sistema maior, como eletrodomésticos e automóveis \cite[p. 1]{BARR:2006}.

\subsection{Microcontroladores}

O microcontrolador consiste em um computador completo em um chip, contendo um processador, memória e interfaces de entrada/saída em um sistema integrado. Essa arquitetura integrada permite criar soluções compactas e de baixo custo para aplicações específicas, sendo o cérebro de grande parte dos sistemas embarcados onde o controle de dispositivos é necessário, atuando no processamento de dados de sensores ou automatizando tarefas \cite{VALDESPEREZ:2013}.

\subsubsection{ESP32}

O microcontrolador ESP32 é um sistema em um chip (SoC, do ingles \textit{System on a Chip}) de baixo custo desenvolvido pela Espressif Systems, projetado para uma vasta gama de aplicações, desde redes de sensores de baixa potência até tarefas complexas, sendo um componente central na prototipagem e desenvolvimento de projetos de Internet das Coisas (IoT) \cite{ESPRESSIF:2016}. Trata-se de um chip híbrido que integra conectividade WiFi (802.11 b/g/n) e Bluetooth em um único módulo \cite{ESPRESSIF:ESP32DS}.


\section{Internet das Coisas} %ok

A Internet das Coisas (IoT, do ingles \textit{Internet of Things}) pode ser compreendida como uma rede na qual objetos físicos podem formar conexões entre si. Esses objetos podem ser equipados com sensores, atuadores e módulos de comunicação sem fio, interagindo entre si e com servidores centrais, formando um sistema inteligente e versátil, o qual pode ser aplicado em diversos setores, como agricultura, indústria, logística e automação residencial \cite{ATZORI20102787}.

\section{Medição de Energia Elétrica} 

A medição de energia elétrica é relevante para o controle do fluxo de energia nos sistemas de distribuição. Sendo assim, na corrente alternada, a energia é constituída por três elementos fundamentais que compõem a potência elétrica: potência ativa, potência reativa e potência aparente.

A potência ativa é a parte da energia que pode ser convertida em trabalho útil (calor ou luz) e que é medida em quilowatts (kW). A potência reativa se expressa em quilovolt-ampères reativos (kVAR) e é aquela que requer energia para criar e manter, em equipamentos indutivos (motor, transformador, dentre outros), os campos magnéticos de que necessitam para operar. A potência aparente (kVA) é a combinação vetorial entre a potência ativa e reativa e representa a potência fornecida pela rede \cite{WILDI:2013}.

O faturamento de energia elétrica do setor residencial é feito apenas em função da potência ativa consumida. Já a potência reativa, apesar de estar presente em todos os consumidores, em diversos graus de magnitude, é faturada somente para os consumidores de alta tensão, sobre os quais serão cobradas multas para o consumo de potência reativa acima do limite \cite{ANEEL:2000}.

\subsection{Transformadores} %ok

O transformador é um dispositivo eletromagnético estático que opera com base no princípio da indução eletromagnética, permitindo a transferência de energia elétrica entre dois ou mais circuitos sem conexão elétrica direta. Sua principal função é elevar ou reduzir os níveis de tensão em sistemas de corrente alternada (CA), mantendo inalterada a frequência do sinal \cite[p. 189]{WILDI:2013}. Conforme ilustrado na Figura~\ref{fig:transIdeal}, o enrolamento primário é caracterizado pela corrente $i_p(t)$ , tensão $V_p(t)$  e número de espiras $N_p$ , enquanto o enrolamento secundário é definido por $i_s(t)$ , $V_s(t)$  e $N_s$ , respectivamente.

\begin{figure}[!htb]%% Ambiente figure
     %\captionsetup{width=0.55\textwidth}%% Largura da legenda
     \caption{Desenho esquemático de um transformador ideal}%% Legenda
     \label{fig:transIdeal}%% Rótulo
     \includegraphics[scale=0.45]{transformador_basico_Chapman}%% Dimensões e localização
     \fonte{Adaptado de \cite{CHAPMAN:2013}}%% Fonte
     \addcontentsline{loge}{figure}{\protect\numberline{\thefigure}Desenho esquemático de um transformador ideal}
     %\addcontentsline{loge}{geralgraph}{\protect\numberline{\thegeralgraph}Fluxograma do Firmware}
 \end{figure}
\FloatBarrier

Estruturalmente, o transformador é composto por duas ou mais bobinas condutoras enroladas sobre um núcleo comum de material ferromagnético. Em geral, essas bobinas não apresentam conexão elétrica direta entre si, sendo o acoplamento realizado exclusivamente através do fluxo magnético compartilhado no interior do núcleo.

O enrolamento conectado à fonte de energia elétrica é denominado enrolamento primário ou de entrada, enquanto o enrolamento responsável por fornecer energia à carga é denominado enrolamento secundário ou de saída. Quando presente, um terceiro enrolamento recebe a designação de enrolamento terciário, sendo empregado em aplicações específicas, conforme a finalidade do transformador \cite[p. 66]{CHAPMAN:2013}.

\subsubsection{Transformadores de corrente}

São transformadores especializados utilizados em série com um circuito elétrico para medir com precisão a corrente alternada que flui pelo circuito. Devido à natureza de seu uso, como medição e proteção de sistemas, a potência desses transformadores costuma ser pequena, normalmente abaixo de 200 VA \cite[p. 189]{WILDI:2013}.
A variação com núcleo partido, ilustrada na Figura~\ref{fig:transUsado}, é particularmente interessante, pois pode ser facilmente instalada em um sistema existente sem necessidade de desconectar a linha ou interromper o fornecimento de energia \cite{SIEMENS:2015}.


\begin{figure}[!htb]%% Ambiente figure
     %\captionsetup{width=0.55\textwidth}%% Largura da legenda
     \caption{Transformador de corrente de nucleo bipartido}%% Legenda
     \label{fig:transUsado}%% Rótulo
     \includegraphics[scale=0.1]{transformador_nucleo_bipartido}%% Dimensões e localização
     \fonte{Autoria Própria}%% Fonte
     \addcontentsline{loge}{figure}{\protect\numberline{\thefigure}Transformador de corrente de nucleo bipartido}
          %\addcontentsline{loge}{geralgraph}{\protect\numberline{\thefigure}transformador de corrente de nucleo bipartido}
 \end{figure}

\section{Medição do Fluxo de Água} %usar fotos das peças que eu tenho

%A medição e o controle do fluxo de água são processos significativos para a gestão dos recursos hídricos, e por isso, vários métodos e ferramentas foram desenvolvidos para tratar estes processos, apresentando diferentes níveis de viabilidade conforme a utilização.

A medição e monitoramento do fluxo de água são processos importantes no gerenciamento de recursos hídricos, o que levou ao desenvolvimento de diversos métodos e ferramentas para tratar desses processos, os quais apresentam diferentes graus de viabilidade de acordo com a aplicação.

Para superfícies aquáticas como os rios e riachos, têm-se métodos como o medidor de molinete, o método do flutuador, a técnica dos vertedores e sensores ultrassônicos de nível, que estimam a vazão com base na velocidade da corrente e na área da seção transversal do curso d'água \cite{COSTA_Camargo_Tolentino_Akutsu_Periotto_Tanaka:2023}.

Já em ambientes controlados nos quais os líquidos são transportados por encanamentos, como em sistemas de abastecimento e esgoto, podem ser aplicados outros métodos, como hidrômetros mecânicos, sensores eletromagnéticos, ultrassônicos \cite{flow:2016}.

\subsection{YF-S201}

O sensor de vazão de água YF-S201, ilustrado na Figura~\ref{fig:YF-S201}, é composto por uma turbina com um ímã acoplado e um sensor de efeito Hall colocado externamente. Quando a água passa pelo sensor, ela faz a turbina girar, movimentando o ímã. Esse movimento gera um sinal no sensor de efeito Hall, que produz pulsos elétricos cuja frequência é proporcional à vazão de água que passa pelo sensor. Esses pulsos elétricos podem ser lidos e interpretados por um sistema de controle, como um microcontrolador, para estimar o volume ou a taxa de fluxo de água.

\begin{figure}[!htb]%% Ambiente figure
     %\captionsetup{width=0.55\textwidth}%% Largura da legenda
     \caption{Sensor YF-S201}%% Legenda
     \label{fig:YF-S201}%% Rótulo
     \includegraphics[scale=0.09]{figuras/YF-S201.jpg}%% Dimensões e localização
     \fonte{Autoria Própria}%% Fonte
     \addcontentsline{loge}{figure}{\protect\numberline{\thefigure}Sensor YF-S201}
 \end{figure}
