\chapter{FUNDAMENTAÇÃO TEÓRICA}

Neste trecho são apresentados os pilares teóricos sobre os quais o trabalho é estruturado.

\section{Sistemas Embarcados} %ok

Um sistema embarcado é uma combinação integrada de hardware e software, sendo projetada para uma função especifica. Em contraste com os computadores de uso pessoal, que atuam executando uma ampla gama de tarefas, os sistemas embarcados são criados de forma a executarem uma tarefa em especifico, sendo frequentemente utilizados como parte de um sistema maior, como eletrodomésticos e automóveis.\cite[p. 1]{BARR:2006}

\subsection{Microcontroladores} %ok

O microcontrolador consiste em um computador completo em um chip, contendo um processador, memória e interfaces de entrada/saída em um sistema integrado. Essa arquitetura integrada permite criar soluções compactas e de baixo custo para aplicações específicas, sendo o cérebro de grande parte dos sistemas embarcados onde o controle de dispositivos é necessário, atuando no processamento de dados de sensores ou automatizando tarefas.\cite{VALDESPEREZ:2013}

\subsubsection{ESP32} %ok

O ESP32 é um SoC de baixo custo desenvolvido pela Espressif Systems, projetado especificamente para uma vasta gama de aplicações, desde redes de sensores de baixa potência até tarefas complexas, sendo um componente central na prototipagem e desenvolvimento de projetos de Internet das Coisas (IoT). \cite{ESPRESSIF:2016}

Trata-se de um chip híbrido que integra conectividade Wi-Fi (802.11 b/g/n) e Bluetooth em um único módulo. \cite{ESPRESSIF:ESP32DS}


\section{Internet das Coisas} %ok

A Internet das Coisas (IoT) pode ser compreendida como uma rede na qual objetos físicos podem formar conexões entre si. Esses objetos podem ser equipados com sensores, atuadores e módulos de comunicação sem fio, interagindo entre si e com servidores centrais, formando um sistema inteligente e versatil, o qual pode ser aplicado em diversos setores, como agricultura, indústria, logística, e automação residencial. \cite{ATZORI20102787}

\section{Medição de Energia Elétrica} %parece bom
%A medição de energia elétrica é relevante para o controle do fluxo de energia nos sistemas de distribuição de energia. Sendo assim, a mesma, na corrente alternada, é constituida por três elementos fundamentais que compõem a potência elétrica: potência ativa (kW), potência reativa (kVAR) e potência aparente (kVA).
A medição de energia elétrica é relevante para o controle do fluxo de energia nos sistemas de distribuição. Sendo assim, na corrente alternada, a energia é constituida por três elementos fundamentais que compõem a potência elétrica: potência ativa (kW), potência reativa (kVAR) e potência aparente (kVA).

A potência ativa é a parte da energia que pode ser convertida em trabalho útil (calor ou luz) e que é medida em quilowatts (kW). A potência reativa se expressa em quilovolt-ampères reativos (kVAR) e é aquela que requer energia para criar e manter, em equipamentos indutivos (emotor, transformador), os campos magnéticos que necessitam para operar; a potência aparente (kVA) é a combinação vetorial entre a potência ativa e reativa, e representa a potência fornecida pela rede\cite{WILDI:2013}.

O faturamento de energia elétrica do setor residencial é feito apenas em função da potência ativa consumida. Já a potência reativa, apesar de estar presente em todos os consumidores, em diversos graus de severidade, é faturada somente para os consumidores de alta tensão, sobre os quais serão cobradas multas para o consumo de potência reativa acima do limite\cite{ANEEL:2000}.


%A medição de energia elétrica é importante para o monitoramento do fluxo de energia em sistemas de distribuição. Na corrente alternada (CA), a potência elétrica é constituida por três componentes fundamentais: potência ativa (kW), potência reativa (kVAR) e potência aparente (kVA). 

%A  potência ativa é a parcela da energia que pode ser convertida em trabalho útil (como calor ou luz) e é medida em quilowatts (kW). A potência reativa, expressa em quilovolt-ampères reativos (kVAR), está associada à energia necessária para criar e manter campos magnéticos em equipamentos indutivos (como motores e transformadores), não realizando trabalho útil, mas é necessária para que esses dispositivos funcionem. Já a potência aparente (kVA) é a combinação vetorial das potências ativa e reativa, representa a potência total fornecida pela rede. \cite{WILDI:2013}

%O faturamento de energia elétrica no setor residencial é feito somente com base na potência ativa consumida. A potência reativa, apesar de estar presente em todos os consumidores, com diversos graus de severidade, só é levada em consideração para consumidores de alta tensão, dos quais multas serão cobradas caso o consumo de potência reativa exceda o limite. \cite{ANEEL:2000}

\subsection{Transformadores} %ok

O transformador é um dispositivo eletromagnético estático que transfere energia elétrica entre dois ou mais circuitos através da indução eletromagnética. Ilustrado na Figura~\ref{fig:transIdeal} Sua função principal é alterar o nível de tensão da corrente alternada (CA), aumentando-o ou diminuindo-o, sem modificar a frequência. \cite[p. 189]{WILDI:2013} 

\begin{figure}[!htb]%% Ambiente figure
     %\captionsetup{width=0.55\textwidth}%% Largura da legenda
     \caption{Desenho esquemático de um transformador ideal}%% Legenda
     \label{fig:transIdeal}%% Rótulo
     \includegraphics[scale=0.5]{transformador_basico_Chapman}%% Dimensões e localização
     \fonte{Adaptado de \cite{CHAPMAN:2013}}%% Fonte
     \addcontentsline{loge}{figure}{\protect\numberline{\thefigure}Desenho esquemático de um transformador ideal}
 \end{figure}
\FloatBarrier

Estruturalmente, o transformador é composto por duas ou mais bobinas condutoras enroladas sobre um núcleo comum de material ferromagnético. Em geral, essas bobinas não apresentam conexão elétrica direta entre si, sendo o acoplamento realizado exclusivamente através do fluxo magnético compartilhado no interior do núcleo.

O enrolamento conectado à fonte de energia elétrica é denominado enrolamento primário ou de entrada, enquanto o enrolamento responsável por fornecer energia à carga é denominado enrolamento secundário ou de saída. Quando presente, um terceiro enrolamento recebe a designação de enrolamento terciário, sendo empregado em aplicações específicas, conforme a finalidade do transformador. \cite[p. 66]{CHAPMAN:2013}

\subsubsection{Transformadores de corrente}

São transformadores de especializados utilizados em série com uma linha de transmissão para medir com precisão a corrente alternada que flui pela rede. Devido a natureza de seu uso, como medição e proteção de sistemas, a potência desses transformadores costuma ser pequena, normalmente abaixo de 200 VA. \cite[p. 189]{WILDI:2013}

A variação com nucleo partido é particularmente interessante pois pode ser facilmente instalado em um sistema existente sem necessidade de desconectar a linha ou interromper o fornecimento de energia. \cite{SIEMENS:2015}

\begin{figure}[!htb]%% Ambiente figure
     %\captionsetup{width=0.55\textwidth}%% Largura da legenda
     \caption{transformador de corrente de nucleo bipartido}%% Legenda
     \label{fig:transUsado}%% Rótulo
     \includegraphics[scale=0.1]{transformador_nucleo_bipartido}%% Dimensões e localização
     \fonte{Autoria Própria}%% Fonte
     \addcontentsline{lof}{figure}{\protect\numberline{\thefigure}transformador de corrente de nucleo bipartido}
 \end{figure}

\section{Medição do Fluxo de Água} %usar fotos das peças que eu tenho

%A medição e o controle do fluxo de água são processos significativos para a gestão dos recursos hídricos, e por isso, vários métodos e ferramentas foram desenvolvidos para tratar estes processos, apresentando diferentes níveis de viabilidade conforme a utilização.

A medição e monitoramento do fluxo de água são processos importantes no gerenciamento de recursos hídricos, o que levou ao desenvolvimento de diversos metodos e ferramentas para tratar desses processos, as quais apresentam diferentes graus de viabilidade de acordo com a aplicação.

Para superfícies aquáticas como os rios e riachos, tem-se métodos como o medidor de molinete, o método do flutuador, a técnica dos vertedores e sensores ultrassônicos de nível, que estimam a vazão com base na velocidade da corrente e na área da seção transversal do curso d'água \cite{COSTA_Camargo_Tolentino_Akutsu_Periotto_Tanaka:2023}.

Já em ambientes controlados nos quais os líquidos são transportados por encanamentos, como em sistemas de abastecimento e esgoto, podem ser aplicados outros métodos, como hidrômetros mecânicos, sensores eletromagnéticos, ultrassônicos \cite{flow:2016}.

\subsection{YF-S201}

O sensor de vazão de água YF-S201, ilustrado na Figura~\ref{fig:YF-S201} é composto por uma turbina com um ímã acoplado e um sensor de efeito Hall. O fluxo de água por dentro do sensor faz com que a turbina gire, o que movimenta o imã que por sua vez aciona o sensor de efeito Hall, o que, por fim, gera um numero de pulsos elétricos proporcionais ao fluxo de água que podem ser lidos e interpretados por um sitema de controle, como um microcontrolador.

\begin{figure}[!htb]%% Ambiente figure
     %\captionsetup{width=0.55\textwidth}%% Largura da legenda
     \caption{sensor YF-S201}%% Legenda
     \label{fig:YF-S201}%% Rótulo
     \includegraphics[scale=0.1]{figuras/YF-S201.jpg}%% Dimensões e localização
     \fonte{Autoria Própria}%% Fonte
     \addcontentsline{lof}{figure}{\protect\numberline{\thefigure}sensor YF-S201}
 \end{figure}

%\section{Custo do chuveiro elétrico} %tenho que esplicar o porque ele é ruim em algum lugar %movido para a justificativa

%No Brasil, o chuveiro elétrico é um dos aparelhos eletrodomésticos mais comuns nas residências, de acordo com \cite{EPE:PDE2031:2022}, é presente em mais de 70\% dos lares. Sua popularidade se deve ao seu baixo custo e facilidade de instalação, que dispensa a complexa infraestrutura exigida por sistemas a gás ou solar. No entanto, esse aparelho é também um dos maiores responsáveis pelo consumo residencial de energia elétrica.

%O consumo de energia elétrica do chuveiro é o terceiro maior em uma residência típica, ficando atrás apenas do ar-condicionado e da geladeira. Em média, o chuveiro elétrico responde por cerca de 14\% do consumo total de energia elétrica doméstica \cite{EPE:PDE2031:2022}. Um banho de 15 minutos em um chuveiro de 5,5 kW, por exemplo, consome cerca de 1,375 kWh, o que, multiplicado pelo valor médio da tarifa residencial no Brasil (aproximadamente R\$ 0,78/kWh, em 2025 \cite{ANEEL:RankingTarifas:2022}), representa um custo em torno de R\$ 1,07 por banho — valor que pode dobrar em regiões com bandeiras tarifárias vermelhas ou escassez hídrica. 

%Além do impacto causado por seu alto consumo energético, o uso do chuveiro elétrico também está diretamente ligado ao consumo de água, de acordo com \cite{MARZALL_NASCIMENTO:2023} pode ser atribuido a 25,89\% da água consumida por uma residencia. Um chuveiro convencional pode consumir entre 4 e 6 litros de água por minuto \cite{NORTEAQUECEDORES:2018}, o que significa que um único banho de 15 minutos pode gastar até 180 litros de água. Esse volume representa uma parcela significativa do consumo diário per capita, estimado em 148,2 litros \cite{SNSA:SNIS_AE_2023}. 

%Portanto, o chuveiro elétrico exerce uma dupla pressão sobre os recursos domésticos: eleva os custos com energia elétrica e contribui substancialmente para o consumo de água potável.


%\section{Psicologia da Conservação} %%talvez
%A Psicologia da Conservação
%O eco-feedback (ou feedback ecológico) 
%Complementarmente, a teoria do nudge (termo cunhado por Thaler e Sunstein, 2008)
