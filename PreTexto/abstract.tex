%%%% ABSTRACT
%%
%% Versão do resumo para idioma de divulgação internacional.

\begin{abstractutfpr}%% Ambiente abstractutfpr
The increasing cost of water and energy resources, combined with the global need for sustainability, highlights the importance of conscious consumption in the domestic environment. Often, the lack of real-time information about spending during daily activities, such as showering, prevents the adoption of more economical and ecological habits. Given this scenario, the main objective of this work was the development of a low-cost embedded system, applying Internet of Things (IoT) concepts for the detailed monitoring of consumption in electric showers. The methodology involved the integration of an ESP32 microcontroller with a water flow sensor and a non-invasive current transformer, allowing for the precise collection of data on the volume of water and electrical energy used. The collected information is processed locally by the device, which calculates the duration, consumption, and total cost of each shower, storing a detailed history in its internal memory. As a result, a functional prototype was obtained that provides real-time statistics and consumption history through an intuitive web interface, accessible by any device on the local network, as well as on an attached LCD. It is concluded that the developed system is an effective tool for resource conservation in a smart home, providing the user with the necessary knowledge to understand their financial and environmental impact, thus encouraging conscious consumption and waste reduction.
\end{abstractutfpr}
