%%%% ABSTRACT
%%
%% Versão do resumo para idioma de divulgação internacional.

\begin{abstractutfpr}%% Ambiente abstractutfpr
With the increasing costs of water and energy resources, combined with the global need for sustainability, it becomes evident the importance of conscious consumption in the domestic environment. Such conscious consumption is hindered by the lack of real-time information about the expenses of daily tasks, such as bathing, which leads to less adoption of more economical and sustainable habits. Given this scenario, the present work has as its main objective the development of a low-cost embedded system for monitoring the costs of showers with electric water heaters, providing immediate feedback to the user. This objective was achieved through the use of sensors connected to a microcontroller, which processes the readings and provides the calculated data to the user. After field tests, with the system installed in a house bathroom, the real-time feedback system on expenses proved efficient in discouraging prolonged showers, therefore, it succeeded in promoting conscious consumption and reducing waste.
%The increasing cost of water and energy resources, combined with the global need for sustainability, highlights the importance of conscious consumption in the domestic environment. Often, the lack of real-time information about spending during daily activities, such as showering, prevents the adoption of more economical and ecological habits. Given this scenario, the main objective of this work was the development of a low-cost embedded system, applying Internet of Things (IoT) concepts for the detailed monitoring of consumption in electric showers. The methodology involved the integration of an ESP32 microcontroller with a water flow sensor and a non-invasive current transformer, allowing for the precise collection of data on the volume of water and electrical energy used. The collected information is processed locally by the device, which calculates the duration, consumption, and total cost of each shower, storing a detailed history in its internal memory. As a result, a functional prototype was obtained that provides real-time statistics and consumption history through an intuitive web interface, accessible by any device on the local network, as well as on an attached LCD. It is concluded that the developed system is an effective tool for resource conservation in a smart home, providing the user with the necessary knowledge to understand their financial and environmental impact, thus encouraging conscious consumption and waste reduction.
\end{abstractutfpr}
