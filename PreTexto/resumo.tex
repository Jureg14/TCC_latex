%%%% RESUMO
%%
%% Apresentação concisa dos pontos relevantes de um texto, fornecendo uma visão rápida e clara do conteúdo e das conclusões do
%% trabalho.

\begin{resumoutfpr}%% Ambiente resumoutfpr
O crescente custo dos recursos hídricos e energéticos, somado à necessidade global de sustentabilidade, evidencia a importância de um consumo consciente no ambiente doméstico. Muitas vezes, a falta de informação em tempo real sobre o gasto durante atividades cotidianas, como o banho, impede a adoção de hábitos mais econômicos e ecológicos. Diante deste cenário, o presente trabalho teve como objetivo principal o desenvolvimento de um sistema embarcado de baixo custo, aplicando conceitos de Internet das Coisas (IoT) para o monitoramento detalhado do consumo em chuveiros elétricos. A metodologia empregada envolveu a integração de um microcontrolador ESP32 com um sensor de fluxo de água e um transformador de corrente não invasivo, permitindo a coleta precisa de dados sobre o volume de água e a energia elétrica utilizados. As informações coletadas são processadas localmente pelo dispositivo, que calcula a duração, o consumo e o custo total de cada banho, armazenando um histórico detalhado em sua memória interna. Como resultado, foi obtido um protótipo funcional que disponibiliza as estatísticas em tempo real e o histórico de consumo através de uma interface web intuitiva, acessível por qualquer dispositivo na rede local, e também em um display LCD acoplado. Conclui-se que o sistema desenvolvido é uma ferramenta eficaz para a conservação de recursos em uma casa inteligente, fornecendo ao usuário o conhecimento necessário para compreender seu impacto financeiro e ambiental, incentivando assim o consumo consciente e a redução de desperdícios.
\end{resumoutfpr}

