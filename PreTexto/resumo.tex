%%%% RESUMO
%%
%% Apresentação concisa dos pontos relevantes de um texto, fornecendo uma visão rápida e clara do conteúdo e das conclusões do
%% trabalho.

\begin{resumoutfpr}%% Ambiente resumoutfpr 
%%AO MODIFICAR ISSO, MODIFIQUE O ABSTRACT
Com os custos crescentes dos recursos hídricos e energéticos, somado à necessidade global de sustentabilidade, torna-se evidente a importância de um consumo consciente no ambiente doméstico. Tal consumo consciente é dificultado pela ausência de informação em tempo real sobre os gastos de tarefas cotidianas, como o banho, o que leva a uma menor adoção de hábitos mais econômicos e sustentáveis. A partir desse cenário, o presente trabalho tem como principal objetivo o desenvolvimento de um sistema embarcado de baixo custo para monitoramento dos custos de banhos com chuveiros elétricos, fornecendo feedback imediato ao usuário. Tal objetivo foi alcançado por meio do uso de sensores ligados a um microcontrolador, o qual processa as leituras e fornece os dados calculados ao usuário. Após testes em campo, com o sistema instalado no banheiro de uma casa, o sistema de feedback em tempo real dos gastos se mostrou eficiente em desincentivar banhos prolongados, portanto, teve sucesso em promover um consumo consciente e a redução de desperdícios.

\end{resumoutfpr}

